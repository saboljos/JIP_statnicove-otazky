\documentclass[a4paper, 11pt]{article}

\usepackage[version=3]{mhchem}
\usepackage{siunitx}
\usepackage{graphicx}
\usepackage{amsmath}
\usepackage[total={16cm,25cm}, top=3cm, left=2.5cm, includefoot]{geometry}
\usepackage{pgfplots}
\usepackage{float}
\usepackage{amsfonts}
\usepackage{booktabs}
\usepackage{easytable}
\usepackage{mathtools}
\usepackage{comment}
\usepackage[makeroom]{cancel}
\renewcommand{\figurename}{Obrázek}
\renewcommand{\refname}{Zdroje}
\renewcommand{\tablename}{Tabulka}
\renewcommand{\contentsname}{Obsah}

\setlength\parindent{0pt}
\setcounter{section}{0}
\renewcommand{\labelenumi}{\alph{enumi}.}
\usepackage{xcolor}
\definecolor{light-gray}{gray}{0.95}
\newcommand{\code}[1]{\colorbox{light-gray}{\texttt{#1}}}
\newcommand{\logoCVUT}{\includegraphics[width = 0.5\textwidth]{img/symbol_cvut_konturova_verze_cb.pdf}}
\newcommand{\logoFJFI}{\includegraphics[width = 0.4\textwidth]{img/fjfi_logo.png}}

\begin{document}
% \include{veličiny}
% \include{zkratky}

% titulní strana
\thispagestyle{empty}

\begin{center}
	{\LARGE
		České vysoké učení technické v Praze \par
		Fakulta jaderná a fyzikálně inženýrská
	}
    \vspace{10mm}

    \begin{tabular}{c}
		\textbf{Katedra jaderných reaktorů} \\[3pt]
    \end{tabular}

   \vspace{10mm} \logoCVUT \vspace{15mm}

   {\huge \textbf{Jaderné inženýrství v praxi}\par}
   \vspace{5mm}
   {\huge \textbf{Magisterské studium}\par}

   \vspace{15mm}
   {\Large \MakeUppercase{Státnicové otázky}}

   \vfill
   {\large
    \begin{tabular}{ll}
    Rok: & 2025
    \end{tabular}
   }
\end{center}

% Prohlášení
\newpage
\thispagestyle{empty}

%~
\vfill

\vspace{1em}

Ahojky, dostávají se ti do rukou státnicové otázky ze předmětu Jaderné inženýrství v praxi. Prosím, měj na paměti, že celkový přehled není v některých tématech ucelený a že se v textu můžou vyskytovat (a stoprocentně vyskytují) chyby jak gramatické, tak faktické. Ale i přestože text má k dokonalosti hodně daleko, předávám ho dál a věřím, že třeba někomu pomůže :)))\\
\\
T.K.\\
\\
No nazdar, jestli tenhle text čteš, tak se ti do rukou dostaly opravené, obohacené, vylepšené a hlavně aktuální otázky dle okruhů na magisterské SZZ v roce 2024. Opět platí to samé co nahoře, je tam spousta chyb, ale snad by měl text poskytovat intenzivnější vhled do problematiky.\\
\\
J.M. + Š.J.\\
\\
Budeme rádi za jakékoli doplnění, opravu nebo komentář a držím palce ke státnicím!!!

\vspace{2em}

\clearpage{\pagestyle{empty}}

\include{none}

\newpage
\parskip=0pt
\begin{small}
\tableofcontents
\end{small}
\parskip=7pt
\newpage

\section[Rentgenofluorescenční analýza]{Energiově a vlnově dispersní rentgenfluorescenční analýza}


\section[Aplikace rentgnofluorescenční analýzy]{Zpracování spekter při použití rentgenfluorescenční analýzy, kvalitativní a kvantitativní analýza, matricové jevy}

\subsection{Kvalitativní analýza}

Nevím přesně o co jde, ale čekal bych to, že při detekci nebudu detekovat jen jednotlivé píky, ale i spojitý kus spektra. Ten je způsoben průchodem elektronu skrze prostředí, které ho zpomaluje, čemuž se říká brzdné záření. Jednotlivé píky jsou pak přisuzovány detekovaným prvkům ve vzorku zkoumaného materiálu.

Podle velikosti energie, popř. vlnové délky, je zjištěno, co je to za prvek, neboť energie fluorescenčního RTG záření je tzv. charakteristické RTG záření a je specifická pro daný prvek. Následně pomocí MCA jsou dle energie rozděleny jednotlivé píky, přičemž jejich amplituda/velikost odpovídá intenzitě, neboli kvantitě. Tím je zajištěno kvalitativní i kvantitativní analyzování vzorku.

V případě vlnově disperzní RFA tak analyzuji většinou jen jednu vlnovou délku, takže tam mám kvalitativní analýzu na ten jeden prvek a kvantita odpovídá velikosti píku.

\subsection{Kvantitativní analýza}

Závisí na: 

\begin{itemize}
    \item obsahu prvku ve vzorku,
    \item intenzitě primárního fotonového záření (čím vyšší tok primárních fotonů, tím vyšší fluorescenční tok),
    \item vlnové délce/energii primárních fotonů (je vhodné, aby energie primárního záření byla jen o málo větší než vazebná energie elektronů ve slupce, neboť tehdy je nejvyšší účinný průřez pro fotoefekt $\rightarrow$ to znamená, že pro různě těžké prvky musím mít různé ozařovací zdroje):
    \item[-] Pro lehké prvky se dá využít RTG lampa a radionuklidy s měkkým zářením (Fe-55, Cm-244).
    \item[-] Pro středně těžké prvky: Am-241.
    \item[-] Pro těžkě prvky (Au, W, U, Pb, apod.): Co-57, Cs-137, Ce-144.
    \item RF výtěžek (fluorescenční výtěžek + výtěžek Augerových elektronů),
    \item ztráty záření,
    \item matrice vzorku = matricové jevy (dají se potlačit, kompenzovat a nebo za určitých podmínek zanedbat).
\end{itemize}

Kalibrace pro kvantitativní analýzu většinou vyžaduje mít k dispozici alespoň jeden referenční materiál známého složení.

Koncentrace ve vzorku zkoumaného materiálu a neznámého složení se stanoví přímým srovnáním s měřením ref. materiálů (trojčlenka). Tento způsob ale nebere v potaz matricové jevy.

\subsection{Matricové jevy}

Matricové jevy představují to, jak nám do naměřené koncentrace daného prvku zasahuje vliv koncentrace ostatních prvků, neboli vliv matrice.

Výrazně se projevují při měření kovů a velmi málo, nebo vůbec, se neprojeví u měření tenkých vrstev, či při měření malých koncentrací zkoumaného prvku v dané matrici. Matricové jevy mohou mít za následek snížení, nebo zvýšení signálu, který pak neodpovídá skutečné koncentraci.

\textbf{Zvýšení signálu:}
\begin{itemize}
    \item Primární zesílená excitace = excitovaný prvek A emituje čáry, které mohou excitovat jiný prvek B, a já pak detekuji vyšší koncentraci prvku B. Podmínkou je, aby $Z_A > Z_B$.
    \item Sekundární zesílená excitace = excitovaný prvek A zesíleně excituje prvek B, a ten zase prvek C. Dostávám pak zesílenou excitaci prvku C. Podmínkou je, aby: $Z_A > Z_B > Z_C$.
\end{itemize}

\textbf{Snížení signálu:}
\begin{itemize}
    \item Zeslabení primárního excitačního záření.
    \item Zeslabení fluorescenčního záření.
    \begin{itemize}
        \item Zeslabené charakteristického záření prvku A absorpcí na prvku B. Pod podmínkou, že $Z_A > Z_B$.
    \end{itemize}
\end{itemize}

\subsection{Metody kompenzace Matricových jevů}

\textbf{Absolutní metody kompenzace:}

Používají se pro energiově disperzní XFA/RFA, kdy je známo celé spektrum vzorku $\rightarrow$ dopředu vím, co vzorek obsahuje.

Stanovení obsahu prvku na základě výpočtu z intenzity spektrální čáry charakteristického RTG záření daného prvku. K tomu musím znát:

\begin{itemize}
    \item Energetické a intenzitní složení spektra,
    \item Fluorescenční výtěžek,
    \item Zeslabovací koeficient pro primární i sekundární zážení,
    \item Přístrojové konstanty.
\end{itemize}

\textbf{Metoda alfa koeficientů:}

Pokud by se matricové jevy neuplatňovaly, koncentrace prvků by se stanovovaly snadno. Např. koncentrace by byla přímo úměrná intenzitě charakteristického záření. Změnami složení matrice je však intenzita charakteristického záření zkoumaného prvku ovlivněna (absorpcí záření, nebo zesilujícími jevy).
	
	$C_i \sim  N_i$ - bez matricových jevů
	
	$C_i \sim N_i(1+\sum_{j}\alpha_{ij}C_j)$ - s korekcí
	
	$\rightarrow$ Koeficient $\alpha_{ij}$ vyjadřuje, jak prvek matrice $j$ ovlivní stanovení koncentrace analytu $i$, určovány experimentálně $\rightarrow$ vyžaduje to velké množství referenčních materiálů.

\textbf{Zřeďovací metoda:}

Přidání většího množství látky o známém složení a zeslabení záření ve vzorku je dáno vlastnostmi známé přidané látky. 

Podle toho, jestli přidávaná látka zeslabuje více či méně, tak ji tam dám méně či více. Metoda je vhodná pro látky, které lze obecně dobře homogenizovat (např. rozpustné látky), a také pro takové látky, které samy neprodukují charakteristické záření (např. voda).

Tato metoda sice snižuje vliv matricových jevů, ale také klesá intenzita charakteristického RTG záření, které chceme měřit.

\textbf{Metoda vnitřního standardu:}

Přidání známého množství prvků nebo sloučeniny do zkoumaného vzorku, avšak přidávané prvky nesmí být totožné jako ty, co jsou ve stanovovaném vzorku. 

Vhodný vnitřní standard je $Z-1$ nebo $Z+1$ oproti stanovovanému prvku (prvek s podobnými absorpčními a excitačními vlivy jako má zkoumaný prvek se přidává ke vzorku, poté se měří se poměr intenzit).

Nevýhodou je obtížné stanovování v případě velkého počtu prvků ve vzorku.

\textbf{Metoda přídavku standardu:}

Přidání známého množství prvku, který má být ve vzorku stanoven. Tím, že to přidám, se mi zvýší signál/intenzita. Interpolací naměřených hodnot k nulové intenzitě charakteristického RTG záření pak dostávám zápornou hodnotu koncentrace tohoto prvku. Tím mohu stanovit kvantitu před tím, než jsem tam přidal něco navíc.

\textbf{Metoda tenké vrstvy:}

Vzorek ve velmi tenké vrstvě $\rightarrow$ předpokládá se zde eliminace rušivých vlivů matrice.

\textbf{Metoda Comptonova rozptylu:}

Odhadnutí absorpčních vlastností matrice na základě intenzity rozptýleného záření z Comptonova jevu.

Fotoelektrický jev je silně závislý na $Z$ prvku, zatímco Comptonův jev není. Proto látka s nízkým $Z$ produkuje více rozptýleného záření a jeho intenzita je přibližně nepřímo úměrná hmotnostnímu součiniteli zeslabení.
\section[Elektronová mikrosonda]{Elektronová mikrosonda}


\section[Aplikace ionizujícího záření v geologii a geofyzice]{Aplikace ionizujícího záření v geologii a geofyzice}

\subsection{Radiometrické datování hornin}

\textbf{Uran -- olovo:}

= Výhradně pro vulkanické horniny, k datování se používá Zr $\rightarrow$ ZrSiO$_4$.

\begin{itemize}
    \item[1)] Máme zirkon ZrSiO$_4$, který obsahuje malé množství U-238 a U-235.
    \item[2)] Dochází k jejich rozpadu $\rightarrow$ vznik stabilních izotopů olova.
    \item[3)] Zirkon krystalizuje z magmatu $\rightarrow$ silně vytlačuje veškeré olovo -$\rightarrow$ předpoklad: po vyvření není ve vyvřelině žádné olovo $\rightarrow$ uzavřený systém pro U i Pb.    
    \item[4)] Jakékoliv olovo, co je detekováno, poté odpovídá rozpadu uranu (tzv. radiogenní olovo).
    \item[5)] Určí se poměr U/Pb $\rightarrow$ hmotnostní spektrometrie .
\end{itemize}

\textbf{Burial dating method -- Al a Be:}

\begin{itemize}
    \item[1)] Máme kosmogenní izotopy $^{26}$Al a $^{10}$Be -- vznik interakcí kosmického záření -- akumulace v horninách jako křemen
    \item[2)] Vznikne zhrubaa konstantní poměr $^{26}$Al a $^{10}$Be .
    \item[3)] Po pohřbení $\rightarrow$ odstínění od kosmického záření $\rightarrow$ poločas hliníku: 729 000 let, u beryllia: 1,39 milionů let $\rightarrow$ hliník se rozpadá rychleji .
    \item[4)] Z poměru můžu určit stáří.
    \item[5)] Limitace na látky obsahující křemen, stáří v stovkách tisíc až milionech let.
\end{itemize}

\textbf{Letecké monitorování} = spektrometrické měření záření gama ze svrchní vrstvy půdy. Umožňuje získat informace o vlastnostech hornin a půd a zejména o jejich obsahu přírodních radionuklidů (především K, U, Th).

Můžu využít i při vyhledávání ložisek uranu, měření v dolech. Dále lze gamma spektrometrii využít v závodě na zpracování uranu, kde jsou jednotlivé vozíky/haldy uranu rozřazovány podle intenzity záření.

Ve stavebnictví využívám při měření radonu. Ten kromě jiného může být měřen při vyhledávání zlomů v půdě, neboť cestuje vzhůru právě skrz zlomy.

\subsection{Jaderná karotáž (well logging)}

Karotáž je metoda v geofyzice, která se používá k měření fyzikálních vlastností hornin v podzemí, zejména při průzkumu a těžbě ropy a zemního plynu. Radiometrická karotáž je jedním z typů této metody, která využívá radioaktivních vlastností hornin.

Princip spočívá v detekci a měření radioaktivního záření emitovaného horninami. Toto záření pochází z přirozeně se vyskytujících radioaktivních prvků, jako jsou uran, thorium a K-40, které jsou součástí hornin. Tyto prvky emitují gama záření, které lze detekovat a měřit pomocí vhodných detektorů umístěných na vrtacích zařízeních.

Při radiometrické karotáži se na vrtací nástroje (karoty) umístí detektory, které registrují gama záření vysílané z hornin. Intenzita tohoto záření může poskytnout informace o složení hornin. Například vysoký obsah uranu může indikovat přítomnost uranových rud, což může být důležité pro průzkum ložisek uranu.

\textbf{Nuclear borehole logging} (jaderná karotáž) je metoda, která využívá velké prostupnosti neutronů a gama záření. Využívá se v průmyslu pro ropu, zemní plyn a uran. Jedná se o metody, které umožňují detekci nestabilních izotopů, a nebo metody, které takovéto izotopy vytváří, což se pak detekuje. Výhodou je dobrá penetrace záření, díky čemuž záření snadno projde skrze obalové materiály. Tyto metody je možné využívat i pokud je vrt vyplněn kapalinou. 

Metody:

\begin{itemize}
    \item \textbf{Gama karotáž}: Nejvíce využívané, jedná se o pasivní metodu. V zásadě jen přijímáme měření. Nejčastější aplikace je v lithologii (nauka o výzkumu a popisu usazených hornin) a v stratigrafii (určování stáří hornin). Zaznamenává se celkové gama záření detekované ve vrtu, neboli expoziční příkon v hornině, či stanovení obsahu jednotlivých radioaktivních prvků v rudním průzkumu. 
    
    Hlavní radioizotopy: thorium (Th-230 a Th-232), draslík (K-40) a uran (U-235 a U-238).
    
    Pokud signál je zesílený $\rightarrow$ asi břidlice, pokud zeslabený $\rightarrow$ pískovec, vápenec, dolomit -- můžu mít ještě spektrální gamma karotáž, čímž stanovím obsah radioaktivních prvků ve vrtu.

    \item \textbf{Gama-Gama karotáž}: Jedná se o aktivní metodu, kde je potřeba učinit nějakou "akci". Gama záření je vyzářeno ze zdroje v zavedené sondě ve vrtu. Toto záření pak prochází okolníma šutrama, interaguje, a pomocí Comptonova rozptylu je zpětně detekováno v detektorech, které se taktéž nachází v zavážené sondě. Výsledkem je stanovení hustoty, pórovitosti atd.
    
    Platí totiž, že odezva bude nepřímo úměrná elektronové hustotě. Detektor v karotážní sondě měří intenzitu gama záření, které se vrací k sondě po interakci s horninou. Nižší intenzita detekovaného záření indikuje vyšší hustotu horniny, protože více záření bylo rozptýleno nebo absorbováno. Naopak vyšší intenzita naznačuje nižší hustotu horniny. 
    
    Druhá možná varianta je detekce sekundárního záření vzniklé fotoefektem, čímž se stanovuje obsah těžkých prvků jako Ba, Sb, Pb. Zdrojem gama záření je často Cs-137. Detektory jsou vhodně odstíněny od zářiče.

    \item \textbf{Neutronová karotáź}: Obsahuje neutronový zdroj v zavedené sondě a detektory pro záznam interakcí, ke kterým dochází v blízkosti vrtu, do kterého je sonda zavedená. Emitované vysokoenergetické neutrony jsou postupně zpomalovány (nejvíce na jádrech vodíku) a následně mohou být absorbovány v materiálu a nebo v detektoru neutronů. Proto lze hovořit o \textbf{neutron-gama karotáži} (měřím charakteristické gamma vzniklé při interakci neutronu s horninou) a o \textbf{neutron-neutron karotáži}, kdy detekuji zpomalené tepelné neutrony, které difundovaly až k detektoru. Většina těchto interakcí závisí na množství přítomného vodíku, a tedy i vody v šutrácích, které jsme provrtali při vrtání vrtu. Nejčastější zdroj vysokoenergetických neutronů je AmBe.

    Neutronová karotáž je obecně souhrn pro neutron-neutron karotáž (zdroj rychlých neutronů $\rightarrow$ zpomalení v hornině $\rightarrow$ stanovení pórovitosti hornin (zpomalení na vodíku)). Dále sem patří neutron-gama karotáž (radiační záchyt neutronů, stanovení pórovitosti, rozhraní plyn-ropa a plyn-voda a také některých prvků Cl, Ni, Fe, Cu, Ti, Mn) a na závěr sem patří Neutronová aktivační karotáž (v podstatě aktivační analýza a měříme charakteristické záření vzniklého radioaktivního izotopu $\rightarrow$ stanovení Cu, Mn, Al, Si, F).
    
% V geologickém průzkumu se už dávno využívá tzv. radioaktivní karotáž. při ní se do geologického vrtu nejprve spustí
% sonda s neutronovým zářičem a poté se měří sekundární radioaktivita geologických vrstev, vyvolaná tokem neutronů.
% Měřením aktivity plynných radionuklidů v půdě se určuje stáří geologických vrstev, rozptylem neutronů se měří vlhkost
% půdy nebo přítomnost zdrojů podzemní vody či ropy

\end{itemize}

Vodohospodáři využívají radionuklidy k měření průtoků v řekách i vodovodních potrubích. Do vody nebo vodního toku je vstříknut radioizotop (bez zátěže ŽP) a následně je tento izotop detekován v místě s různými detektory (můžou být různě podél břehu i v hloubce). Na základě času a koncentrace lze stanovit průtok. Používám, když nemohu použít standardní metody. 

Rozlišuji:

\begin{itemize}
    \item metodu časové distribuce -- sleduji za jak dlouho se mi vzorek dopravil z bodu A do bodu B a v jaké koncentraci,
    \item metodu ředění -- podle toho, jak se koncentrace v toku mění, určím průtok.
\end{itemize}

Ozařováním je možno ošetřit také odpadní vody obsahující některé nebezpečné látky ještě před přivedením do běžných čističek odpadních vod. Zářiče s radiokobaltem zabraňují množení nežádoucích mikroorganizmů, které snižují kvalitu pitné vody ve studních.

\begin{figure}[H]
    \centering
    \includegraphics[width=0.8\linewidth]{img/Neutronová karotáž.png}
    \caption{Neutronová karotáž}
\end{figure}

\subsection{Metody využívající zpětný odraz gama}

\textbf{Stanovení obsahu popela v uhlí:}

Zpětný rozptyl záření gama se například využívá pro stanovení obsahu popela v uhlí, kde využívá energetické závislosti píku zpětně odraženého gama záření. Princip je vlastně identický jako u gamma-gamma karotáže:

\begin{itemize}
    \item[1)] Emitované fotony interagují s materiálem (uhlí) $\rightarrow$ Comptonův rozptyl, část fotonů odražena zpět.
    \item[2)] Detektorem měřím intenzitu odražených fotonů.
    \item[3)] Míra rozptylu závisí na hustotě elektronů v materiálu, což je ovlivněno atomovým složením a hustotou.
    \item[-] uhlí -- především H, C, O, N a S \& popel $\rightarrow$ těžší sloučeniny -- oxidy křemíku, hliníku, železa, vápníku etc.
    \item[-] popel -- větší $Z$ $\rightarrow$ více elektronů \& větší hustota $\rightarrow$ větší elektronová hustota $\rightarrow$ pravděpodobnější rozptyl a absorpce $\rightarrow$ pokles intenzity.
    \item[4)] Naopak tam, kde je hodně uhlí a málo popela $\rightarrow$ vysoká intenzita odraženého gamma.
\end{itemize}

Řízení přívodu uhlí do elektrárny:

\begin{itemize}
     \item Provoz elektrárny vyžaduje přívod uhlí o specifické výhřevnosti. V případě jakékoliv odchylky může dojít ke ztrátě výkonu nebo dokonce k výpadku kotle.
     \item Obsah popela uhelného uhlí a výhřevnost je sledována před plněním sila, aby byla zajištěna dostatečná kvalita uhlí pro správný provoz kotle.
     \item  V případě, že on-line bilance kvality uhlí neodpovídají požadavkům kotle, lze do některých sil nakládat uhlí různé kvality.
    \item  Parametry veškerého uhlí, které je pro kotel k dispozici, slouží obsluze k úpravě pracovních parametrů kotle podle potřeby.
\end{itemize}

\textbf{In-situ analýza a kontrola těžby:}

V některých případech je možné gamma záření využít pro in-situ analýzu hornin během těžby. Speciální zařízení mohou měřit intenzitu gamma záření přímo na místě a poskytovat okamžitou zpětnou vazbu o kvalitě rudy. To umožňuje optimalizovat proces těžby a zpracování nerostů tím, že těžba může být řízena na základě skutečné kvality vytěženého materiálu.

\begin{itemize}
    \item In-situ leaching: roztok kyseliny rozpouští uran z rudy $\rightarrow$ následně je monitorována koncentrace uranu v roztoku, optimalizováno pro co nejmenší dopasd na ŽP a efektivnost práce.
    \item Lze také při klasické těžbě - monitoruji jednotlivé "vozíky" s rudou.
\end{itemize}

%Nejsem si jistý, jestli se měří radiaktivní izotopy (uran, thorium ...) -- tzn. pasivní metoda, nebo se sleduje také rozptyl (nebo charakteristické záření?) asi by šlo sestrojit různé metody.

Pásová analýza (nebo-li analýza materiálu v průběhu toho co mi běží na běžícím páse a mohu tak online stanovovat kvalitu uhlí): Metoda stanovení obsahu popela v uhlí pomocí gama záření využívá principu měření intenzity gama záření, které prochází vzorkem uhlí. Popel v uhlí má specifické vlastnosti, které ovlivňují absorpci a rozptyl gama záření. Na základě těchto změn lze určit obsah popela ve vzorku.

\textbf{Analýza kalů (slurry analysis):} 

Gama záření se používá k měření koncentrace pevných částic v kalu. Princip je podobný jako u měření obsahu popela v uhlí či stanovování hustoty, ale také se využívá při loggingu (karotáž, viz Vrtání do země, psáno výše).

\underline{Pozn.:} Jak odhalím podíl prvku ve sloučenině? Pokud je prvek dostatečně těžký, resp. všechny prvky, co jsou ve sloučenině, není nic jednoduššího než použít RFA a využít detekci charakteristického RTG záření. Pokud je ovšem třeba jen jeden prvek ze dvou, tak to jde taky tak udělat a od celkového množství odečtu to, co naměřím, čímž získám zbytek. Možností je také využít rozdílu mezi fotoefektem a comptonovým roztpylem, kde fotoefekt má vyšší pravděpodobnost pro těžší atomy, a proto se na celkovém podílu reakcí více podílí pro lehké atomy Comptonovův rozptyl. Poslední možností je, pokud to umožňuje, využít neutronovou aktivační analýzu.
\section[Využití iontových svazků v materiálovém výzkumu]{Využití iontových svazků v materiálovém výzkumu: Základní typy urychlovačů, Metody RBS, kanálování, PIXE, PIGE, ERDA a NRA}


\section[Využití jaderně-fyzikálních metod v materiálovém výzkumu]{Využití jaderně-fyzikálních metod v materiálovém výzkumu: Mössbauerova spektrometrie, elektron-pozitronová anihilační spektroskopie, neutronová aktivační analýza}

\subsection{Mossbauerova spektrometrie}

Byla objevena německým fyzikem Rudolfem Mossbauerem v roce 1958. Je založeno na rezonanční fluorescenci $\gamma$ záření

Obvykle používané prvky: $^{57}$Fe, $^{119}$Sn, $^{121}$Sb, $^{129}$I.

Široká škála aplikací při studiu chemických vazeb, anorganické chemii pevných látek, atd. Hlavní množství článků a studií je ale zaměřeno na železo.

Jedná se o metodu izotopicky citlivou (bezodrazová jaderná rezonanční absorpce/fluorescence gama záření).

\subsubsection{Princip}

Mossbauerův jev = bezodrazová jaderná rezonanční absorpce gamma záření

\begin{itemize}
    \item Založeno na emisi a absorpci $\gamma$ záření emitovaného jádrem bez zpětného rázu.
    \item Atomy ve zdroji emitující $\gamma$ záření musí být stejné, jako v atomy ve vzorku.
    \item Vlastně je to tak, že jádra ve vzorku jsou excitovaná gammou, která je emitovaná při stejné deexcitaci ve zdroji.
    \item Zde zdroje fotnů jsou generovány fotony o dané energii a pokud je tato energie veeeelmi přesná jako je energie mezi vzbuzeným a základním stavem atomu či jádra ve zkoumaném materiálu, tak dojde k tzv. rezonanční absorpci.
    \item Poté dochází opět k přechodu zkoumaného jádra do základního stavu emisí fotonu o té samé energii, co to původně vyvolala
    \item Celé co chceme je dosáhnout překrycu emisní a absoprpční čáry, což je velmi složité neboť je šířka píku (energie) velmi malá a komplikuje to celé jev tzv. zpětného rázu.
\end{itemize}

\subsubsection{Zpětný ráz}
\begin{itemize}

\item Při emisi vysokoenergetické částice z jádra funguje zpětný odraz, část hybnosti je předaná emitujícímu jádru (analogicky u zbraně -- taky mě to hodí trochu dozadu).
\item Energie gamma záření má proto o trošku nižší energii než je energie rezonance, a to o energii tohoto zpětného rázu.
\item Pro energii zpětného rázu platí: $E_{R} = \dfrac{E_{\gamma}}{2 M c^2}$, kde $M$ je hmotnost emitujícího jádra a $E_{\gamma}$ je energie emitovaného fotonu.
\item S narůstající hmotností $M$ klesá energie zpětného odrazu, a proto platí, že jeli energie emitovaného gama záření malá oproti hmotnosti jádra, jež jej emituje, pak se energie zpětného rázu pohltí a s jádrem to nehne. V opačném případě je třeba tuto energii nějak zohlednit a ideálně ji kompenzovat.
\item V případě nevázaného jádra je posun energie značný a není možné Mossbauerův jev pozorovat. Neboli nedojde k překryvu emisní a absorpční čáry.

Příkladem je známé železo Fe-57 u něhož je šířka čáry 4,6 neV a energie zpětného rázu je 2meV. To znamená, že emitované gama o velikosti 14.4 keV, které na něj letí s cílem ho excitovat, tak mi je k ničemu, protože energie fotonu je nižší o ty 2 meV a tudíž nedosáhnu překryvu emisní a absorpční čáry, která je ultra tenounká. Nutno dále zmínit, že o energii zpětného rázu se mi navíc posouvá nejen emisní čára, ale i absorpční čára, neboť to má vliv i na jádro, jež energii přijíma. Souhrně řečeno: U volného jádra (plynné či kapalné médium) dochází při emisi k zpětnému rázu, o který se sníží energie emitovaného fotonu a pak nevystačuje energie na excitaci cílového jádra.

\item Zvýšení hmotnosti jádra, aby byla energie zpětného rázu pohlcena/kompenzována lze dosáhnout uvázáním jádra do krystalické mřížky $\rightarrow$ zpětný ráz je daleko menší, do hmotnosti se taky připočítávají okolní vázané atomy, neboli mřížka tlumí odraz.
\end{itemize}


\textbf{Jak to funguje:}

\begin{itemize}
    \item Mám zdroj, který emituje přesně energii, kterou chci pro excitaci atomu/jádra v mém vzorku (pro tyto účely uvažujme Fe-57).
    \item Ve vzorku mám atomy/jádra Fe-57, které toto záření absorbují, jsou excitována a poté při deexcitaci emitují gama záření o té samé energii, která je excitovala z mého zdroje (jak bylo řečeno výše, tak zdrojem je to samé jádro, které chci ve vzorku excitovat. Toto jádro, které já beru jako zdroj tak excituju nějakým externím RN zdrojem například).
    \item Emitované záření ze vzorku je pak emitované do celého prostorového úhlu.
\end{itemize}

\begin{figure}[H]
    \centering
    \includegraphics[width=\linewidth]{img/Mossbauer excitace jádra.png}
    \caption{Mossbauer excitace jádra}
\end{figure}

\subsubsection{Pravděpodobnost jevu}

Celá pravděpodobnost, že nastane tento jev je popsána skrze účinný průřez rezonanční absorpce závisející na spinu základního a excitovaného jádra, vlnové délce fotonů a na šířce čáry.

%- účinný průřez rezonanční absorpce 

$$\sigma_0 = \dfrac{\lambda^2}{2\pi} \cdot \dfrac{2I_e + 1}{2I_g +1} \cdot \dfrac{1}{1+\alpha},$$

kde $I_{g,e}$ jsou jaderné spiny základního a rezonančního stavu, $\alpha$ koeficient vnitřní koverze a $\lambda$ vlnová délka fotonu.
 
 % - pak je tam vzorec pro výpočet počtu přechod mezi $E-E_0$ a $E-E_0 + dE$, poocí f-faktoru
 
Dále hraje roli, tzv f-faktor, který popisuje pravděpodobnost bezfononového procesu. Tento faktor závisí na silách v mřížce a tedy čím více mřížka vibruje, tím více bude f-faktor klesat. Pravděpodobnost Mossbauerova jevu tedy roste se snižující se energií gama fotonu (menší zpětný odraz), roste se snižující se teplotou (menší tepelný pohyb) a roste se zvyšující se Debeyeovou teplotou (popisuje míru síly vazeb mezi jádrem a okolní mřížkou, resp. obecně míru síly vazeb).
 
 %- pro ideální experiment musí platit, že výchylka atomů je malá v  porovnání s vlnovou délkou gamma záření, teplota je menší než Debeyova teplota, energie přechodu není příliš vysoká (< 150 keV) a energie zpětného rázu je malá
 
Mossbauerovou spektormetrii lze měřit pouze krystalické a amorfně tuhé látky, zamrznuté roztoky (enmohu kapaliny a plyny).

\subsubsection{Uspořádání experimentu}

Základní součástí experimentu je zdroj záření, zkoumaný vzorek a detektor. Ne každý izotop je však vhodný pro zkoumání a měření touto metodou (již řečeno).

Aparatura se skládá z:

\begin{itemize}
    \item Zdroj/zářič: Nejvíce využívaný je izotop Fe-57, dále je Sn-119 či Au nebo Eu. Hlavní je ale to železo.
        \begin{itemize}
            \item Omezení zkoumání vzorků na ty, které obsahují např. právě to Fe-57
            \item Energie fotonu musí být v určitém rozmezí energií, a proto to nefunguje pro všechny jádra (nad 180 keV je energie zpětného rázu moc velká a pod 5 keV se neprojevuje rezonanční absorpce).
            \item Potřebuji zdroj fotonů s velmi přesnou energií aby to fungovalo
            \item Požadavky na zdroj jsou tedy: Dostupnost, trvanlivost, šířka čáry
        \end{itemize}
    \item Absorbátor/zkoumaný vzorek: Pokud je vzorek moc tlustý pak dojde k úplné absorpci a nebudu mít žádný signál. Příliš tenký vzorek je ale také špatně, avšak existuje vztah pro výpočet efektivní tloušťky (pro železo je to 1-5 mg/cm2 Fe atomů).
    \item Detektor: V zásadě lze využít širokou škálu detektorů
        \begin{itemize}
            \item Scintilační NaI(Tl) -- nejvíce využívaný scintilák, pak existuje ještě na bázi Ytria, ale ten má na hovno energietické rozlišení a účinnost, ale má rychlou odezvu a je dobrý pro vysoké četnosti. Obecně jsou scintiláky dobré, jelikož jsou rychlé.
            \item Proporcionální (plynem plněný) -- lepší energetická rozlišovací schopnost (je to pravda?).
            \item Polovodičový -- na bázi Si či Ge, avšak pro tuto metodu je to možná až moc overkill a zbytečně drahé.
        \end{itemize}

    \item Pojezd zdroje -- Jelikož je překryv energií hodně malý, tak mohu jemnou modulací energie pomocí Dopplerovské modulace doladit, aby došlo k překryvu absorpčního a emisního píku. V praxi to provedu tak, že zdroj umístím na nějakou membránu a pohybuji s ním tam a zpět (kmitám) - lze si představit jako membrána u reproduktoru.
\end{itemize}

\textbf{Rozdělujeme 3 uspořádání při experimentu:}

Měřenní může v obecnosti trvat hodiny až dny nebo týdny, přičemž podmínky testu jsou statické. Mimo jiné mohu na zkoumaný materiál aplikovat nějaké vnější vlivy = magnetické pole, změna teploty apod.

\begin{itemize}
    \item Transmisní = Zdroj -- vzorek -- detektor. Měřím co je za vzorkem. Měřím tzv. emisní spektrum, protože měřím, to co mi ze vzorku vylétává

    \item Konverzní = Měřím jednak emitované gama záření, ale i RTG či konverzní elektrony, augerovy elektrony, čímž dostávám informace z různé hloubky, ovšem celkově je to stále jen desítky mikrometrů.

    \item Odrazová = Detektor je mimo osu primárního svazku a tím mám geometrii na odraz
\end{itemize}


\begin{figure}[H]
    \centering
    \includegraphics[width=0.8\linewidth, trim={1.5cm 11cm 1.5cm 11cm}, clip]{img/mossbauer_instrumentation.pdf}
    \caption{Experimentální uspořádání transmisní Mossbauerovy spektrometrie}
    \label{fig:2_6_mossbauer_zapojeni}
\end{figure}

%- můžu mít transmisní geometrii, odrazovou geometrii, můžu měřit konverzní elektrony
%- CEMS (Conversion Electrons Mossbauer Spectroscopy), TMS (Transmission Mossbauer Spectrometr), CXMS (Conversion X-ray Mossbauer Spectroscopy)
%- vzorek nesmí být ani moc tlustý, ani moc tenký - definuji efektivní tloušťku
%- jako detektory se využívají scintilárky, proporcionální detektory, polovodičové detektory nebo Si-PIN detektory
%- scintilační detektory NaI(Tl), YAP(Ce) 
%- proporcionální dteektory mají lepší rozlišení v porovnání se scintilákama
%

\subsubsection{Využití}

\begin{itemize}
    \item Mossbauerovské jádro mi funguje v mém materiálu jako sonda a říká mi informace o jeho okolí = lokální mikrostruktura
    \item Sledování hyperjmených interakcí Mossbauerovského jádra s jeho okolím = jedná se o interakce, které nějaký způsobem ovlivňují excitační stavy - Energetické stavy budou ovlivňovány svým okolím
    \item Mohu zkoumat vliv např. externího magnetického pole na posun či rozštěpení excitovaného či základního stavu (excitovaný/základní stav mají nějakou hladinu a buď může dojít vlivem externích jevů k jejich posuvu a nebo rozštěpení na více hladit)
    \item Informace o charakteru vazeb
    \item Informace o spinu
    \item Informace o oxidačním stavu
    \item Kvalitativní (identifikace sloučenin) i kvantitativní (poměrové zastoupení) např. fáze železa a jejich poměrové zastoupení i uspořádání. Interakce konkrétního nuklidu se svým okolím. 
    \item Využívá se především v materiálovém výzkumu a dále geologie, farmaceutický průmysl, biologie/medicín. Spíš ale prostě ten materiálový výzkum a je to.
\end{itemize}

\subsubsection{Hyperjemné interakce}

Jedná se o interakce, které nějaký způsobem ovlivňují excitační stavy, a to jak posunem nebo štěpením na více hladin. Bavíme se o:

\begin{itemize}
   \item elektrické monopólové interakci -- izomerický posun
   \item elektrická kvadrupólová -- kvadrupólové štěpení
   \item magnetická dipólová -- magnetické hyperjemné štěpení
\end{itemize}

\begin{figure}[H]
   \centering
   \includegraphics[width=0.8\linewidth, trim={2cm 10cm 2cm 10cm}, clip]{img/mossbauer_interactions.pdf}
   \caption{Hyperjemné interakce}
   \label{fig:6_2_mossbauer_hyperjemne_interakce}
\end{figure}

\textbf{Elektrická monopólová interakce:}

\begin{itemize}
\item interakce rozložení náboje jádra s hustotou elektronů v prostoru jádra 
\item izomerní posun $\delta = \dfrac{2\pi}{5} Ze^2 [R_e^2 - R_g^2] \cdot {\rho_a - \rho_s}$
\item pro excitovaný, resp. základní stav platí, že poloměr jádra se mění, stejně jako hustota
\item určuji s ohledem na referenční materiál (bcc-Fe)
\item udává informacee o charakteru vazeb, spinu, oxidačním čísle, elektronegativitě
\end{itemize}

\textbf{Elektrická kvadrupólová interakce:}

\begin{itemize}
\item interakce mezi jádrovým kvadrupólovým momentem a nehomogenitami elktrického pole
\item kvadrupólové štěpení: $\Delta = \dfrac{1}{2} eV_{zz} (1+\dfrac{1}{3}\eta^2)^{1/2}$ 
\item jaderná podmínka: elektrický kvadrupólový moment -- $eQ \neq$ 0 ($I$ > 1/2)
\item elektronová podmínka: gradient elektrického pole od elektronů $\neq$ 0 (příspěvek od mřížky a valenčních elektronů)
\item dává informaci o lókální symetrii, oxidačním stavu, charaktere vazeb, spinovém stavu

\end{itemize}
\begin{figure}[H]
   \centering
   \includegraphics[width=0.5\linewidth, trim={4cm 11cm 4cm 11cm}, clip]{img/mossbauer_quadrupole.pdf}
   \caption{Elektrické kvadrupólové štěpení}
   \label{fig:6_2_mossbauer_electric_quadrupole}
\end{figure}

\textbf{Magnetická dipólová interakce:}

\begin{itemize}
   \item interakce magnetického momentu jádra s vnitřním nebo aplikovaným magnetickým polem
   \item $E_{m_1} = - \dfrac{\mu H m_1}{I} = - g_N \beta_N H m_1$
   \item magnetické štěpení hladin jádra (Zeemanův jev)
   \item jádrová podmínka: magnetický dipólový moment $\mu \neq 0$ (I > 0)
   \item elektronová podmínka: intenzita magnetického pole $H \neq 0$
   \item platí výběrová pravidla pro jádrový spin a magnetické kvantové číslo
\end{itemize}

\begin{figure}[H]
   \centering
   \includegraphics[width=0.5\linewidth, trim={5cm 14cm 5cm 14cm}, clip]{img/mossbauer_magnetic_quadrupole.pdf}
   \caption{Magnetické kvadrupólové štěpení}
   \label{fig:6_2_mossbauer_magnetic_quadrupole}
\end{figure}

\subsubsection{Kalibrace}

\begin{itemize}
   \item zdroj záření: $^{57}$Co v matrici Rh, Pd, Cu, Cr
   \item kalibrací ryhlostní stupnice (převedu rychlost na energii)
   \item kalibrační absorbátory - bcc-Fe, $\alpha$-Fe$_2$O$_3$
   \item nastvaení nulové rychlosti
\end{itemize}

\subsubsection{APlikace Mossbauerovy spektrometrie}

\begin{itemize}
   \item strukturní informace, stechiometrie, substituce, nekrystalické systém
   \item identifikace fází
   \item Fe2+ a Fe3+
   \item energetické rozlišení 1 : 10$^{13}$
   \item teplotní a tlakové studie
\end{itemize}

\begin{figure}[H]
   \centering
   \includegraphics[width=0.5\linewidth, trim={2cm 10cm 2cm 10cm}, clip]{img/mossbauer_shrnuti.pdf}
\end{figure}

\subsection{Elektron-pozitronová anihilační spektroskopie}

Udává unikátní informace o defektech v pevných látkách (především vakance, dislokace, shluky vakancí, hranice zrn - principiálně se jedná o oblast se sníženou hustotou kladného náboje -- od jader :D $\rightarrow$ potenciálov jáma pro záchyt pozitronů).

Zdroje pozitronů:

\begin{itemize}
	\item beta+ radionuklidy: jádra bohatá na protony (p $\rightarrow$ n + $\beta$ + + $\nu$), Na-22, Cu-64, Co-58
	\item Produkce e--e+ párů z vysokoenergetických fotonů gama: impuzní zdroj e+
	\item Jaderné reakce: Cd-113(n, gama)Cd-114, Cu-63(n, gama)Cu-64, kontinuální zdroj s vysokou intenzitou
\end{itemize}

$\rightarrow$ Zdroje pro PET: C-11, N-13, O-15, \textbf{F-18}, Ga-68 - výroba: nejčastěji cyklotron

Princip anihilace: pozitronový zářič $\rightarrow$ interakce pozitronu s elektronem $\rightarrow$ vznik dvou fotonů o energii 511 keV (nejpravděpodobnější proce).

\begin{itemize}
    \item anihilace nenastává okamžitě - mezitím: vázaný stav - pozitronium (para a orto pozitronium - podle toho jestli stejný nebo opačný spin pozitronu a elektronu a také podle toho jinak dlouho trvají)
\end{itemize}

\subsubsection{Experimentální techniky}

\begin{itemize}
    \item Doba života pozitronů -- vyzářen pozitron a gama $\rightarrow$ měří se vyzáření pozitornu a gama
    \item Dopplerovo rozšíření -- koincidenční změření energie obou anihilačních fotonů $\rightarrow$ charakterizace chemického okolí defektů
    \item Úhlové korelace
\end{itemize}

\begin{figure}[H]
    \centering
    \includegraphics[width=0.8\linewidth, trim={2cm 10cm 2cm 10cm}, clip]{img/pas_doba_zivota.pdf}
    \caption{Metoda doby života pozitronů.}
    \label{fig:6_2_pas_doba_zivota}
\end{figure}

\begin{figure}[H]
    \centering
    \includegraphics[width=0.8\linewidth, trim={2cm 10cm 2cm 10cm}, clip]{img/dopplerovo_koincidencni_zapojeni.pdf}
    \caption{Metoda Dopplerova rozšíření- koincidenční zapojení}
    \label{fig:6_2_pas_dopplerovo_rozsireni}
\end{figure}

\subsection{Neutronová aktivační analýza}

\begin{itemize}
    \item Je založena na aktivaci chemických prvků přítomných v analyzovaném vzorku. 
    \item Jedná se o jednu z nejvíce citlivých metod chemické analýzy
    \item Obvykle je aplikována s využitím měření následného rozpadu aktivovaných prvků (měření gama)
    \item Existuje i metoda s měřením okamžitého záření a ta je využívána pokud produkt reakce má velmi krátký poločas rozpadu a nebo vzniká stabilní produkt.
    \item Pro NAA se dají vyuźít neutrony všech energií, avšak nejčastěji se využívají tepelné, a to kvůli větší dostupnosti neutronů při této energii a také kvůli energetické závislosti účinných průřezů, které mají v této oblasti vysoké hodnoty.
    \item Množství aktivovaných RA atomů daného prvků ve vzorku je přímo úměrné množství těchto atomů a proto se dá využít i pro kvantitativní analýzu.
    \item Indukovaná aktivita závisí na době ozařování a poločasu rozpadu, přičemž okolo $10x T_{1/2}$ dosahuje akivita saturované hodnoty.
    \item Po aktivaci jader v analyzovaném vzorku následuje měření radioaktivity vzorku a identifikace RN na základě energie a intenzity emitovaného gama záření a s ohledem na poločas rozpadu.
    \item Množství studovaného prvku lze získat z naměřené aktivity s tím, že je nutno brát v potaz:
    \begin{itemize}
        \item Hustotu toku neutronů (resp. energetické spektrum)
        \item Energetickou závislost účinných průřezů
        \item Dobu ozařování
        \item Poločas rozpadu
        \item Detekční účinnost trasy (geometrie, stínění, mrtvá doba atd.)
        \item Dobu měření
        \item Dobu chladnutí (přesun po ozařování k detektoru)
    \end{itemize}
\end{itemize}
 
\textbf{Rozdělení NAA}
\begin{itemize}
    \item Absolutní
        \begin{itemize}
            \item Z přímého měření aktivity umožňuje stanovit množství zkoumaného izotopu, avšak vyžaduje k tomu přesnou znalost neutronového spektra, účinných průřezů s ohledem na energetický rozsah neutronového pole.
            \item Málo kdy využívaný přístup protože přesná znalost neutronového spektra není v praxi obvykle k dispozici
        \end{itemize}
    \item Porovnávací:
        \begin{itemize}
            \item Srovnávání aktivity RN ve zkoumaném vzorku s jeho aktivitou v podobě standardu/etalonu o známé hmotnosti a složení, který byl ozářen za stejných podmínek jako zkoumaný vzorek
            \item Při srovnávací metodě se nevyužívají hodnoty účinných průřezů ani neutronový tok a v případě stejné geometrie při detekci/měření tak ani absolutní detekční účinnost (pro oba měřené je to stejné, tak to není vyžadováno)
            \item Vysoká přesnost, avšak časově náročné pokud vzorek obsahuje vícero prvků, protože pro každý je nutné mít etalon zvlášť.
        \end{itemize}
    \item $k_0$ metoda
        \begin{itemize}
            \item využívá $k_0$ faktory, které se stanovují na základě jaderných dat v kombinaci s experimentálním stanovením.
            \item nezávisí na neutronovém toku a charakteristikách detektoru
        \end{itemize}
    \item Instrumentální NAA -- toto známe a děláme na KJR.
        \begin{itemize}
            \item Nedestruktivní metoda pro stanovení vícero prvků v rámci jednoho měření
            \item Nežádoucím efektem je vzájemné ovlivňování prvků = Vznik jednoho RN reakcemi na dvou různých prvcích (26Mg(n;$\gamma$)27Mg a 27Al(n;p)27Mg) = Tento problém je ovšem řešitelný v případě, kdy jeden z prvků produkuje i další radioaktivní nuklidy a dále také pokud je rozdíl v energetické závislosti účinných průřezů na energii neutronů, tak se dá použít vhodný filtr jako např Cd.
            \item V praxi se měří tak, že se vzorek ozáří v reaktoru na saturovanou hodnotu, pak se nechá vychladnout (snížení aktivity) pokud je moc naaktivovaný a pak se odnese do gama spektrometru a měří se gama záření z rozpadu RN. Z naměřeného spektra se poté stanovuje kvantita a kvalita složení materiálu.
        \end{itemize}
\end{itemize}

\subsubsection{Nastavení experimentálních parametrů}
\begin{itemize}

            \item V rámci tohoto měření, tak je vhodné mít odladěnou dobu ozařování (ať to není zbytečně moc a aktivita dlouhodobě žijících RN není moc vysoká)
            \item dobu vymření (chceme minimalizovat aktivitu všech RN kromě toho, který chceme měřit)
            \item Doba měření (měla by být kratší než nějaký významnější pokles aktivity vzorku - pak mi do toho začne hrát roli pozadí a Rn, K a U)
            \item při ozařování dochází ke vzniku vícero aktivačních produktů z jednoho prvku (pro ověření lze měřit dlouho a zaměřit se na ověření přes poločas rozpadu). Popřípadě aktivační produkty obvykkle emitují $\gamma$ fotony o vícero energiích, čímž je možné taky jednoznačně stanovit.
            \item Měřící geometrie (mrtvá doba)
            \item Velikost vzorku: zbytečně velký vzorek představuje riziko samostínění, samoabsorpce gama záření či zbytečně velkou aktivitu nebo problematické manipulace (to platí i pro zbytečně malý vzorek).
            \item Tok primárních částic přímo ovlivňuje úroveň produkované aktivity.
            \item Ve výsledném výpočtu reakční rychlosti, resp. obecně měření je nutné zohlednit několik faktorů = plocha pod píkem při měření, počet částic v látce na počátku, oprava na čistou dobu měření, doba ozařování, rozpad při vymírání, rozpad při měření, Radiační výtěžek (oprava na intenzitu gama přechodu), oprava na efektivitu detektoru pro danou geometrii (detekční účinnost). Korekce na nerovnoměrné ozařování, korekce na samoabsorpci.
        \end{itemize}

\textbf{Využití a aplikace NAA:}
\begin{itemize}
    \item Monitoring životního prostředí
    \item Zajištění jakosti v průmyslu
    \item Hygienické studie
    \item Certifikace referenčních materiálů
    \item Stopové prvky
    \item Kvalita půdy
    \item Analýza uhlí
\end{itemize}
\section[Jaderně-fyzikální metody v nukleární medicíně]{Jaderně-fyzikální metody v nukleární medicíně: gama kamera, CT, PET}

Jedná se o metody založené na využití farmaceutických radionuklidů, a to ve formě absorpce v těle nebo zavedenímm do lidského organismu. Jako detektory jsou většinou využívány scintilátory, proto metody dělíme na:

\begin{itemize}
    \item dynamickou scintilografii -- sleduje časové změny rozložení radionuklidů, např. činnost orgánů,
    \item planární scintilografie -- statická vizualizace.
\end{itemize}

\subsection{Zdroje záření}

V medicíně lze využívat růzzné druhy zdrojů:

\begin{itemize}
    \item radionuklidy 
    \begin{itemize}
        \item $\gamma$ záření z jaderných reakcí, jsou definovány energií a poločasem rozpadu, musí být vyráběny např. na cyklotronech,
        \item $^{99m}$Tc -- $\gamma$-kamery, $^{18}$F -- PET, $^{123}$I -- $\gamma$-kamera, $^{131}$I -- terapeutický zdroj (prostě ti to vysmaží mutující buňky v nádoru), $^{11}$C -- PET, $^{67}$Ga -- $\gamma$-kamera, $^{81}$Rb -- PET\\
        
        \textit{každý ten zdroj se používá na něco trochu jiného, na diagnostiku mozku použijí asi nějaký jiný než na diagnostiku sleziny atd.}
    \end{itemize}
\end{itemize}

\subsection{Diagnostika}

Dělíme podle činnosti:

\begin{itemize}
    \item morfologická (uspořádání, tvar, velikost),
    \item funkční (činnost orgánů),
    \item dynamická (časová závislost).
\end{itemize}

Podle uspořádání:

\begin{itemize}
    \item absorpční -- RTG, CT,
    \item emisní -- PET, gamma kamera, magnetická rezonance,
    \item kombinovaná -- NAA, CT+PET.
    
\end{itemize}

\subsection{Gamma kamera}

\begin{figure}[H]
    \centering
    \includegraphics[width=0.55\linewidth, trim={2cm 10cm 2cm 10cm}, clip]{img/gamma_kamera.pdf}
    \caption{Základní komponenty gamma kamery}
    \label{fig:2_5_gamma_kamera}
\end{figure}

\begin{figure}[H]
    \centering
    \includegraphics[width=0.5\linewidth, trim={2cm 8cm 2cm 8cm}, clip]{img/gamma_kamera_snimek.pdf}
    \caption{Celotělový snímek z gamma kamery}
    \label{fig:5_2_gamma_kamera_snimek}
\end{figure}

Kolimátor:

\begin{itemize}
    \item paralelní mnohokanálový kolimátor $\rightarrow$ ovlivní směr fotonů, geometické zorné pole kamery, prostorov rozptyl, citlivost systému -- výběr dle energie registrovaných fotonů, rozlišení, skenovací hloubky a požadované citlivosti
    \item výsledný obraz -- počet otvorů, průměr otvorů, délka jednotlivých trubic, materiál atd.
\end{itemize}

Výběr radionuklidu:

\begin{itemize}
    \item poločas rozpadu musí mít několik hodin, produkované $\gamma$ musí ít stovky keV, lehko zabudovatelný do farmaka, v nemocnici musí vydržet několik dní
    \item typicky se používá $^{99m}$Tc
\end{itemize}

\subsection{Počítačová tomografie -- CT}

\subsubsection{Princip}

Zzaložená na zeslabení svazku RTG záření (absorpční metoda):

$$ I = I_0 \cdot exp(-\mu\cdot t). $$

CT: mám sadu $j$ řádků a $i$ sloupců, každý detektor sbírá informaci o zeslabení

$$I_{12} = I_{0} \cdot exp(-(\mu_{1} + \mu_{2})\cdot \Delta t)$$
$$I_{34} = I_{0} \cdot exp(-(\mu_{3} + \mu_{4})\cdot \Delta t)$$
$$I_{13} = I_{0} \cdot exp(-(\mu_{1} + \mu_{3})\cdot \Delta t)$$
$$I_{24} = I_{0} \cdot exp(-(\mu_{2} + \mu_{4})\cdot \Delta t)$$

Přes získané intenzity $I_{12}, I_{34}, I_{13}, I_{24}$ jsem schopný zrekonstruovat koeficienty $\mu_{i}$ ve všech blocích. Ve skutečnosti je ale bloků výrazně více

Generované RTG záření prochází tělem, na druhé straně gantry jsou detekován sadou scintilátorů. Získané data tvoří projekci, kompletní sada zeslabení v závislosti na úhlu $\theta$ a vzdálenosti $t$ je pak sinogram. Následně je obraz rekonstruován.

Sinogram je kompletní soubor projekcí, ukazuje grafickou závislost polohy objektu od úhlu projekce.

\begin{figure}[H]
    \centering
    \includegraphics[width=0.4\linewidth,trim={5cm 10cm 5cm 10cm}, clip]{img/ct_scheme.pdf}
    \includegraphics[width=0.5\textwidth,trim={2.5cm 10cm 3cm 10cm}, clip]{img/ct_projection.pdf}
    \includegraphics[width=0.4\textwidth,trim={3cm 7cm 3cm 10cm}, clip]{img/ct_sinogram.pdf}
\end{figure}

Kontrast se hodnotí vůči koeficientu zeslabení pro vodu, jednotky Hounsfield. Pokud budu proces opakovat a posouvat spirálou axiálně podél těla $\rightarrow$ dostanu 3D obraz.

\subsubsection{Základní součástky CT tomorgafu}

\begin{itemize}
    \item gantry,
    \item RTG trubice,
    \item kolimátor a filtr,
    \item detektory,
    \item systém získávání informací o odezvě,
    \item vyšetřovací lůžko.
\end{itemize}

\begin{figure}[H]
    \centering
    \includegraphics[width=0.5\linewidth,trim={3cm 7cm 3cm 7cm}, clip]{img/ct_ilustrace.pdf}
\end{figure}

\subsection{PET} 

Metoda založená na pozitron-elektronové anihilaci. Zdroj pozitronů $\rightarrow$ beta rozpad.

\subsubsection{Základní součástky PET tomografu}

\begin{itemize}
    \item gantry, 
    \item vyšetřovací lúžko,
    \item systém detektorů, 
    \item počítač,
    \item laserové zaměřovače.
\end{itemize}

Zdroj pozitronů: nejčastěji $^{18}$F v deoxyglukóze. Detektory jsou scintilační detektory (NaI(Tl), berylium germaniový, gadolinium-křemíkový, lutecium-křemíkový), počet detektorů udává rozlišení (tisíce). 

Princip je podobný jako u CT, pomocí detekovaných fotonů z anihilace je pro daný úhel vytvořena projekce, kompletní projekce tvoří sinogram:

$$ \text{sinogram $\rightarrow$ rekonstrukční algoritmus $\rightarrow$ finální obraz.} $$

\subsection{Rozlišení obrazu}

Co narušuje obraz:

\begin{itemize}
    \item útlum -- dáno tloušťkou tkáně, fotoefektem, Comptonovým jevem (sekundární gamma je tlumeno),
    \item Comptonův rozptyl,
    \item náhodná koincidence.
\end{itemize}

Často mohu kombinovat CT a PET. Z CT získám anatomickou strukturu, z PET biologické procesy.


\begin{figure}[H]
    \centering
    \includegraphics[width=0.8\linewidth, trim={1cm 11cm 1cm 11cm}, clip]{img/pet_obrazy.pdf}
    \caption{Snímky z CT a PET tomografie}
    \label{fig:2_6_CT_PET_tomografie}
\end{figure}
\section[Využití synchrotronového záření v materiálovém výzkumu]{Využití synchrotronového záření v materiálovém výzkumu: získávání synchrotronového záření a jeho vlastnosti, příklady experimentálních technik}

Synchrotronové záření, někdy také nazývané jako magnetické brzdné záření, je záření vysílané relativistickými elektrony kroužícími v magnetickém poli a vzniká tak při pohybu nabité částice se zrychlením. Dochází tak k uvolňování EM záření. Jelikož je zrychlení vektorové, nemusí docházet ke zpomalování nebo zrychlování nabité částice, ale stačí změna dráhy pohybu a tím se také utváří zrychlení, jež vede na uvolnění části energie do okolí.

Dále platí, že pokud je při změně dráhy pohybu rychlost částice malá, tak emitované záření je uvolňováno izotropně. Pokud je rychlost vysoká (v/c je cca 1), tak je záření soustředěno do kuželu ve směru pohybu částice s určitým úhlem rozevření kužele, jež závisí na Lorentzovu faktoru, který je nepřímo úměrný rychlosti (čím větší rychlost tím menší rozevření kužele).

\subsection{Schéma uspořádání synchrotronu}

Sychrotron je cyklický urychlovač částic, jež se skládá z elektronového děla, lineárního urychlovače pro prvotní zrychlení částice a posléze ze dvou prstenců (urychlující a akumulační). Vývodem akumulačního prstence je trasa vedoucí do koncové stanice, kde je realizován experiemnt.

\begin{figure}[H]
	\includegraphics[width=7cm]{img/synchrotron.png}
	\includegraphics[width=7cm]{img/synchrotron2.png}
\end{figure}

\subsection{Tvorba synchrotronového záření}

Bylo již řečeno v jakých případech a jak vzniká synchrotronové záření, ale jak se to dělá v praxi? V praxi se využívá tzv. vkládacích zařízení, kterými jsou: ohybací magnet (bending magnet), Wigglery a Ondulátory.

\begin{itemize}
    \item \textbf{Ohybací magnet}: slouží k zakřivení dráhy pohybu nabité částice a přitom je tečně k dráze pohybu uvolňováno záření, které se pak dá kolimovat.

    \begin{figure}[H]
        \centering
        \includegraphics[width=0.5\linewidth]{img/ohybací magnet.png}
        \caption{ohybací magnet}
    \end{figure}

    \item \textbf{Wiggler}: zařízení, jež je tvořeno větším počtem dvojic permanentních magnetů, které zapříčiní klikacení dráhy pohybu částice a to způsobuje tvorbu širokého svazku nekoherentního záření. Intenzita záření je úměrná N (počet magnetů).

    \begin{figure}[H]
        \centering
        \includegraphics[width=0.5\linewidth]{img/Wiggler.png}
        \caption{Wiggler}
    \end{figure}
        
    \item \textbf{Ondulátor}: je to to samé co wiggler (opět permanentní magnety), ale magnety jsou slabé a je to asi i delší. Ta částice se tudíž neklikatí tolik a výsledná intenzita emitovaného záření je dokonce úměrná $N^2$, kde N je počet magnetů

    \begin{figure}[H]
        \centering
        \includegraphics[width=0.5\linewidth]{img/Ondulátor.png}
        \caption{Ondulátor}
    \end{figure}
\end{itemize}

\subsection{Vlastnosti synchrotronového záření}

\begin{itemize}
    \item Spojité a velmi široké spektrum (od IR po RTG záření).
    \item Vysoká monochromatizace.
    \item Při využití ondulátorů, je záření koherentní.
    \item Stabilita svazku.
    \item Impulzní emise (záření je emitované časticemi a podle počtu částic je to víc spojité a nebo více impulsní).
    \item Přeladitelnost (můžeme si vybrat energii) $\rightarrow$ více stupňová monochromatizace na monokrystalech Si $\rightarrow$ Monochromátor s vysokým rozlišením (rozptyl cca 1 meV) $\rightarrow$ kolimace a fokusace pomocí Be čočky nebo K-B zrcadlo a tím fokusace až na rozměry 4x10 $\mu m^2$.
    \item Polarizované.
    \item Briliance = kombinace toku, velikost zdroje, divergence svazku = jedná se v zásadě o trochu komplikovanější Intenzitu pro popis elektromagnetického záření. 
\end{itemize}

\textbf{Využití}

\begin{itemize}
    \item Dá se využít pro jaderný rezonanční rozptyl (studium hyperjemných interakcí, vibrační vlastnosti jader, magnetické přechody apod., a to za extrémních vnějších podmínek = magnetické pole, kryo, vysoká teplota apod..).
    \item RTG mikroskopie -- využití RTG svazků s rozlišením v řádu desítek nm a zobrazení tenkých vrstev a povrchů.
    \item Transmisní X-ray mikroskopie (TXM) -- Dobrý kontrast, vysoké rozlišení.
    \item RTG sepktroskopie = měření chemického složení.
    \item RTG absorpční spektroskopie = informace o typu a vzdálenostech sousedních atomů.
    \item RTG tomografie = 3D obrazy drobných objektů s velmi vysokým rozlišením na úrovni mikrometrů.
\end{itemize}

\subsection{XFEL} 

Zařízení XFEL (X-Ray Free Electron Laser) je zařízení, které je v podobě lineárního urychlovače (European XFEL, SwissFEL) a umožňuje lineárně urychlovat elektrony až na energie 17 GeV. Urychlené elektrony pak projdou ondulátorem a vzníká záření jako ze synchrotronu, které má rozsah energií od 0,01 - 20 keV (velmi malá vlnová délka).

$\rightarrow$ Velmi malá krátkost pulsů (cca desítky fs)

$\rightarrow$ O 10 řádů vyšší briliance jako u 3. generace sychrotronů. 

$\rightarrow$ V zásadě jsou dány frekvence na balík (jak často přichází balík elektronů, resp. pulsu), ale pak je ještě samotná frekvence v rámci balíku a proto je krátkost pulsů velmi malá.

\begin{figure}[H]
    \centering
    \includegraphics[width=0.5\linewidth]{img/časová struktura pulsů fotonů.png}
    \caption{časová struktura pulsů fotonů}
\end{figure}

\newpage
\mbox{}
\newpage

\section[Jednotky a veličiny v dozimetrii]{Jednotky a veličiny v dozimetrii, základy legální metrologie, etalony a stanovená měřidla}

Metrologie je věda zabývající se měřením . Mezi základní cíle a úkoly patří:

\begin{itemize}
    \item definice jednotek a jejich realizace pomocí vědeckých metod,
    \item vývoj a udržování etalonů nejvyšší úrovně,
    \item zajištění fungování měřidel ve výrobní sféře,
    \item zajistit správnost měření v úředních nebo obchodních sférách,
\end{itemize}

\subsection{Základy legální metrologie}

Hlavním je zákon č. 505/1990 Sb. o metrologii $\rightarrow$ upravuje práva a povinnosti subjektů a státních orgánů správy pro účely zajištění správnosti a jednotnosti měřidel a měření.
Dále jsou důležité prováděcí vyhlášky k samotnému zákonu, a to sice:

\begin{itemize}
    \item Vyhláška MPO č. 262/2000 Sb., kterou je zajišťována jednotnost a správnost měřidel a měření.
    \item Vyhláška MPO č. 345/2002 Sb., kterou se stanoví měřidla k povinnému ověřování a měřidla podléhající schvalování.
    \item Vyhláška MPO č. 264/2000 Sb., o základních měřících jednotkách a ostatních jednotkách a o jejich označování.
\end{itemize}

Klasifikace měřidel:

\begin{itemize}
    \item \textbf{Etalony} (primární, sekundární, mezinárodní, státní) = objekt či něco jiného, jež obecně slouží k uchování a realizaci dané jednotky (etalon hmotnosti, kde jednotka je kg, tak je snad krystal křemíku, protože umíme přesně určit počet atomů v mřížce).
    \item \textbf{Stanovená měřidla} = Jedná se o měřidla, která MPO (ministerstvo průmyslu a obchodu) vyhláškou stanoví k povinnému ověřování s ohledem na jejich význam (obchod, daně, sankce, tarify, poplatky, medicína, ochrana ŽP, BOZP).
    \item \textbf{Pracovní měřidla} = jedná se o měřidla jež nejsou etalonem ani stanoveným měřidlem.
    \item \textbf{Certifikované a ostatní referenční materiály} = Jedná se o materiály přesně stanoveného a známého chemického složení, které slouží pro ověřování nebo ke kalibraci (často to bývají etalony)
\end{itemize}

Schvalování typů měřidel je proces, při kterém jsou ověřeny metrologické a technické vlastnosti stanovených měřidel, jež vycházejí z technických norem. Zkoušky pro schválení typu stanoveného měřidla obsahuji funkční zkoušky, zkoušky odolnosti proti rušivým vlivům vnějšího prostředí a zkoušky elektromagnetické kompatibility. Výsledkem je certifikát a přidělení značky schválení typu.

Pod pojmem \textbf{návaznost} rozumíme: (definice je na hovno) V zásadě se jedná o to, že vemu např. 8 vah z celého světa, nastavím je pomocí kvalitního etalonu na to, že toto je 1 kg a potom, když něco naměřím já, tak i Karlíkovi z horní dolní můžu věřit, že to má jak já, protože to má nastavený v návaznosti na ten samý etalon a tudíž i v návaznosti na mne.

\textbf{Ověřování} je proces, resp. potvrzení, že měřidlo má požadované metrologické vlastnosti.

\textbf{Kalibrace} je proces, kdy je periodicky upravováno a kontrolováno, že měřidlo měří to co má a případná kalibrace probíhá pomocí certifikovaných a ostatních referenčních materiálu, kterými jsou nejčastěji etalony.

\subsubsection{Organizace}

\textbf{ÚNMZ = Ústav pro technickou normalizaci, metrologii a státní zkušebnictví}

\begin{itemize}
    \item Zastupování ČR ve věci mezinárodních věcí s ohledem na metrologii.
    \item Dohled na činnosti ČMI.
    \item Kontrola dodržování zákona o metrologii.
    \item Schválení a vyhlášení státních etalonů.
    \item Poskytování metrologických expertíz.
\end{itemize}

\textbf{ČMI}

\begin{itemize}
    \item Uchovávání a rozvoj státních etalonů.
    \item Schvalování typů a ověřování stanovených měřidel.
    \item Certifikace referenčních materiálů.
    \item Kalibrační služby.
\end{itemize}

\subsubsection{Veličiny a jednotky}

Existují různé soustavy jednotek, kde nejvíce je rozšířené SI, dále existuje imperiální, americká apod... Stačí znát SI.

V obecnosti jsou základní jednotky SI popsáné v Zákoně č. 505/1999 Sb. o metrologii. Obsaženy jsou dále odvozené jednotky, násobky, díky a jiné povolené jednotky.

Základní jednotky a veličiny SI jsou:

\begin{itemize}
    \item Čas -- s,
    \item Proud -- A,
    \item Svítivost -- cd,
    \item Látkové množství -- mol,
    \item Teplota -- K,
    \item Hmotnost -- kg,
    \item Délka -- m.
\end{itemize}

Odvozené jednotky vyjadřované jednotkami základními jsou např. hustota, objem, plocha, rychlost, ...

Odvozené jednotky se zvláštním označením i názvem jsou např. náboj, síla, aktivita, absorbovaná dávka, ...

Odvozené jednotky vyjádřené jednotkami základními spolu s jednotkami se zvláštním názvem a označením jsou např. Expozice, PDE, dávkový příkon, dávkový ekvivalent, ...

\textbf{Vlastnosti a výhody SI:}

\begin{itemize}
    \item Systematická a mezinárodně uznávaná soustava veličin a jednotek.
    \item Pro každou veličinu zavedena pouze jedna jednotka.
    \item Odvozené jednotky tvořeny součiny mocnin základních jednotek.
    \item Násobky a díly jednotek vyjádřeny předponami.
\end{itemize}

\textbf{Další mimosoustavné jednotky:} Jsou použitelné spolu s SI v rámci specifikovaných oborů a nebo výjimečně a tehdy, jeli definován vztah k SI jednotkám: uzel, dioptrie, angstrom, barn, atm, curie, calorie.

\textbf{Veličiny ve vztahu k atomové a jaderné fyzice:} Aktivita, přeměnová konstanta, poločas rozpadu, barn, hustota toku, fluence, absorbovaná dávka, kerma, expozice, dávkový ekvivalent, efektivní dávka, PDE, ...

\subsection{Jednotky a veličiny v dozimetrii}

\textbf{Expozice:}

Definována pro popis ionizujících účinků fotonů ve vzduchu: $[X] = C \cdot \text{kg}^{-1}$.

Expozice je definována jako podíl celkového náboje d$Q$ iontů jednoho znaménka, jež vznikly při úplném zabrzdění všech elektronů a pozitronů uvolněných fotony v malém objemu vzduchu, a hmotnosti tohoto objemu vzduchu d$m$.

\begin{equation}
    X = \frac{\text{d}Q}{\text{d}m}
\end{equation}

\begin{figure}[H]
    \centering
    \includegraphics[width=0.5\linewidth]{img/expozice.png}
    \caption{expozice}
\end{figure}

\textbf{Absorbovaná dávka}

Vyjadřuje střední energii předanou IZ dané látce o jednotkové hmotnosti: $[D] = \text{Gy}$.

\begin{equation}
    D = \frac{\text{d}\overline{\varepsilon}}{\text{d}m}
\end{equation}

\textbf{Kerma}

Popisuje působení nepřímo ionizujícího záření z hlediska energetických ztrát primárních částic: $[K] = \text{Gy}$.

\begin{equation}
    K = \frac{\text{d}E_k}{\text{d}m}
\end{equation}

\textbf{Dávkový ekvivalent, resp. ekvivalentní dávka}

Jedná se o veličinu jž využívá tzv. faktoru kvality $Q$, kterým je přenásobena absrobovaná dávka a tento faktor kvality pak popisuje vliv daného IZ na tkáň (bez rozdělování o jakou tkáň se jedná): $[H] = \text{Sv}$.

$Q = 1$ pro elektrony, gama a RTG, dále je $Q = 10$ pro protony, neutrony a $Q = 20$ pro nabité ionty, alfa částice, štěpné produkty.

\begin{equation}
    H = Q \cdot D
\end{equation}    


\textbf{Efektivní dávka}

Jedná se o Dávkový ekvivalent násobený dalším faktorem kvality, a to tentokrát tkáňovým faktorem, který zohledňuje ještě dále, jaká část těla schytala to záření, protože každá část těla je jinak náchylná/odolná. To v praxi znamená, že kostní dřeň je snad nejvíce náchylná, zatímco játra nebo oko, dlaň (kůže) není tolik: $[D] = \text{Sv}$.

\begin{equation}
    D \cdot w_r = H, E = \sum_T w_T \cdot H_T
\end{equation}


\section[Využití detektorů v metrologii aktivity]{Využití proporcionálních detektorů a kapalných scintilátorů v metrologii aktivity radionuklidů}

Metrologie radioaktivity radionuklidů je dělena do dvou základních skupin:

\begin{itemize}
    \item Absolutní = Využívá přímé detekce veličiny (Koincidenční metoda, elektrostatická metoda, kalorimetrická metoda, absolutní počítání částic).
    \item Relativní = Vztah mezi veličinou indikovanou a veličinou měřenou (spektrometrie gama, ionizační komora).
\end{itemize}

\subsection{Využití proporcionálních detektorů}

Jedná se o primární metodu měření = vychází z definice veličiny.

Proporcionální detektor je často válcové geometrie a skládá se z anody (tenký drátek z W, Mo, Cu, ocel, Au pokrytí) a katody (tělo detektoru). Díky velkému rozdílu ve velikosti elektrod je mezi nimi velký rozdíl napětí, a to vytváří velmi intenzivní elektrické pole. Pracovním plynem uvnitř detektoru je často Ar, Kr, Xe + zhášecí plyn, což je například Metan nebo propanbutan.

Výstupní impuls na detektoru je úměrný deponované energii. Typická náplň je plyn P-10 (90\% Ar a 10\% CH$_4$).

Důležité je, aby anodový drátek měl konst. průměr a tloušťku. Výhodou proporcionálního detektoru je, že téměř nemá mrtvou dobu a je tedy hodně dobře schopen detekovat dva signály "naráz".

Je provozován v oblasti proporcionality VA charakteristiky plynového detektoru, kdy měřená četnost téměř nezávisí na napětí (pracovní napětí detektoru je zvoleno v této oblasti). Provoz v této oblasti umožňuje rozvoj townsendovy laviny, jež představuje plynové zesílení.

Plynová naplň v detektoru je přítomna za účelem zesílení vstupního signálu = plynové zesílení. Proto je proporcionální počítač vhodný pro detekci záření malých energií, které je pak třeba zesílit právě přítomným plynem (ionizace plynu k zesílení vstupního signálu, a to řádové E2 až E3).

Proporcionální počítač se používá k měření $\alpha$ a $\beta$ záření.

Proporcionální počítače lze využít jako prosté detektory čítání, ale také částečně pro spektrometrii/spektroskopii neboť výstupní signál je úměrný vstupnímu, resp. deponované energii (s uvážením zesílení od plynu).

\textbf{Druhy proporcionálních počítačů:}

\begin{itemize}

    \item Průtokový: 

        \begin{itemize}
            \item Pracovní plyn protéká detektorem o atmosférickém tlaku a díky tomu je snadná výměna vzorků.
            \item Měření $\alpha$ a $\beta$ záření, zejména pak $\beta$.
            \item Lze udělat i 4$\pi$ variantu.
            \item Vhodný pro měření nízkoenergetické $\beta$ a plynných radioaktivních sloučenin i neutronů (pro neutrony je potřeba konverzní materiál).
        \end{itemize}
    
    \item Tlakový: 

        \begin{itemize}
            \item Plynová náplň má vyšší hmotnostní číslo a plyn je natlakován do cca 1,5 MPa.
            \item Vyšší tlak umožňuje dosáhnout vyšší účinnosti detekce záření (Tím, že je to natlakované, tak se tam vejde více plynu a tím se mi zvyšuje pravděpodobnost interakce záření s plynem).
        \end{itemize}

\end{itemize}


\textbf{Korekce:}

Zejména pro interní proporcionální poćítač, kdy je měřený vzorek uvnitř.

\begin{itemize}
    \item Koncový efekt = Na okrajích elektrod je elektrické pole deformované. Lze kompenzovat využitím dvou počítačů různé délky a rozdíl odezvy odpovídá četnosti naměřené ideálním detektorem o délce rozdílu jejich délek.
    \item Stěnový efekt = částice emitovaná v blízkosti stěn nestačí dostatečně ionizovat (oprava na základě měření s několika počítači stejné délky různých poloměrů, vliv klesá s rostoucím tlakem plynové náplně)
\end{itemize}

\subsection{Využití kapalných scintilátorů}

Jedná se o scintilační detektor, kde scintilační látka je tekutá/roztok. Jedná se o roztoky scintilačních látek v organických rozpouštědlech + měřený vzorek, který je v tomto taktéž rozpuštěn (podle druhu měřeného vzorku se používají rozpouštědla jako je toulen, xylen, ethalon).

Kapalné scintilátory se používají pro detekci IZ ($\alpha$ a $\beta$) o vyšších energiích (minimální detekovatelná energie je 3-10 keV). Využívají se pro tzv. metodu TDCR (triple to double coincidence ratio).

Detekční účinnost je dána účinnosti fotokatody daného fotonásobiče (kvalita fotonásobiče).

\subsubsection{TDCR}

Jedná se o koincidenční zapojení tří fotonásobičů, jež funguje na principu poměru trojných a dvojných koincidencí. Tato metoda slouží ke stanovení počítací účinnosti, a to experimentálně bez potřeby kalibračního standardu.

počítací účinnost ($\varepsilon_D$), neboli parametr TDCR (v zásadě detekční účinnost) se určuje na základě srovnání experimentu a modelu $\rightarrow$ v praxi je měřeno TDCR pro různé účinnosti detekce, které mohu měnit defokusací PMT, optickými filtry atd. Tím získávám závislost naměřené aktivity na TDCR (aktivita by měla být nezávislá na TDCR - kritérium správnosti) a dále závislost detekční účinnosti na TDCR (tu detekční účinnost si měním). Ve výsledku srovnáním experimentu a modelu získám účinnost $\varepsilon_D$:

$$ \left( \frac{\varepsilon_T}{\varepsilon_D} \right)_\text{výpočet} = \left( \frac{N_T}{N_D} \right)_\text{měření}. $$

Výsledný počet řešení TDCR parametru se odvíjí od druhu radionuklidu a složitosti jeho kaskády rozpadu.

Výsledná aktivita je získána jako: 

$$ A = \frac{N_D}{\varepsilon_D}. $$

\begin{figure}[ht!]
    \centering
    \includegraphics[width=1\linewidth]{img/tdcr.png}
    \caption{TDCR}
\end{figure}

\section[Koincidenční metoda \& spektrometrie gama]{Koincidenční metoda stanovení aktivity a spektrometrie záření gama jako sekundární metoda měření aktivity}


\section[Metrologie neutronů \& manganová lázeň]{Metrologie neutronů a metoda manganové lázně včetně zpracování výsledků měření a zdrojů chyb a nejistot}

Z hlediska metrologie neutronů je dobré zmínit nejprve veličiny jako je hustota toku neutronů ($\phi$), neutronová fluence ($\varphi$) nebo emise radionuklidového zdroje neutronů ($S$).

\textbf{Hustota toku neutronů $\phi$} = to samé co fluence, ale za 1s.

\textbf{Neutronová fluence $\varphi$} = podíl počtu neutronů jež dopadnou z libovolného směru na malou kouli a plochy jejího příčného průřezu (kolik neutronů mi projde plochou celkově za celý čas měření)

\textbf{Emise} zdroje $S$ (s$^{-1}$) = počet částic emitovaných ze zdroje za jednotku času.

\subsection{Zdroje neutronů}

\textbf{Radionuklidové zdroje:}

\begin{itemize}
    \item ($\alpha$, n), ($\gamma$, n), spontání štépení
    \item ($\alpha$, n) = typicky AmBe s emisí od $10^5 - 10^8$ za sekundu a energií neutronů od 5 do 10 MeV. Dále existuje PuBe
    \item ($\gamma$, n) = SbBe, NaBe nebo $D_2O$. Výhodou tohoto typu zdrojů je produkce takřka monoenergetických neutronů, avšak za cenu nižší energie.
    \item spontání štěpení = v praxi se využívá hlavně a snad jen $^{252}$Cf, jež má cca 3 \% pravděpodobnost spontáního štěpení a zbytek je alfa přeměna.
    \item Mezi hlavní výhody patří relativně nízká cena, dostuponost, transport, malé rozměry, nízké nároky na provoz a údržbu.
    \item Nevýhodou je neproměnné spektrum, nižší emise neutronů, doprovodné gama záření a nemožnost vypnutí zdroje.
\end{itemize}

\textbf{Generátory neutronů:}

Využívájí fúzních reakcí ve formě D-D, D-T či T-T reakcí za vzniku 3-He, 4-He a 4-He. Největší energie reakce je při fúzi D-T. V zásadě se jedná o zásobník s plynem částic, které jsou pak urychlovací trubicí (jak urychlovač) urychlovány a dopadají na terčík.

\subsection{Metoda manganové lážně}

Jedná se o metodu pro standardizaci emise $S$ zdrojů neutronů.

\textbf{Princip:}

\begin{itemize}
    \item Aktivace $^{55}$Mn ve vodném roztoku $MnSO_4$ pomocí neutronů za vzniku $^{56}$Mn
    \item Rozpad $^{56}$Mn s poločasem rozpadu cca 2,5h na železo $^{56}$Fe, elektron a gama.
    \item Mangan se využívá, neboť má vysoký účinný průřez pro absorpci neutronů (tepelné neutrony cca 100 barnů).
    \item Emise neutronového zdroje $S$ je stanovena na základě saturované aktivity $^{56}$Mn, avšak se zohledněním absorpce neutronů na kyslíku, síře, vodíku a dále se musí zohlednit vliv prahových reakcí na jádrech síry a kyslíku ($T$), korekce na neutrony ztracené ve zdroji a dutinách ($C$) a na závěr korekce na únik neutronů z lázně ($L$).
    \item Saturované aktivity je dosaženo po cca 10 poločasech rozpadu, což zde dělá zhruba jeden den.
    \item Následně je odebrán vzorek roztoku z promíchané lázně a stanovení aktivity odebraného vzorku buď pomocí koincidenční metody nebo gama spektrometrií a následně přepočet na celkovou aktivitu lázně pomocí trojčlenky.
    \item Jinou možností měření je vložit detektor přímo do lázně (scintilák nebo GM).
    \item Z naměřené plochy píku stanovíme aktivitu $^{56}$Mn ($A=\frac{P}{\varepsilon t_{live} Y}$), avšak nutno je učinit korekce na přeměnu po dobu přenosu vzorku do spektrometru, na dobu samotného měření (po tyto doby dochází k rozpadu), mrtvá doba detektoru, radiační výtěžek (pravděpodobnost rozpadu tím procesem, který chci měřit).
\end{itemize}

\begin{equation}
    S = A_{Mn} \cdot \frac{[\sigma_{Mn} + \sigma_S + 4\cdot\sigma_O]\cdot N_{Mn} + [\sigma_H + 1/2 \cdot \sigma_O]\cdot N_H}{\sigma_{Mn}\cdot N_{Mn}} = \frac{A_{Mn}}{f}  \rightarrow  S = \frac{A_{Mn}}{f\cdot(1- T - C - L)}
\end{equation}

\begin{equation}
    A_{\text{Mn}} = \dfrac{P\cdot \lambda \cdot \dfrac{t_{\text{real}}}{t_{\text{livey}}}}{(1 - e^{-\lambda \cdot t_1}) \cdot e^{-\lambda \cdot (t_2 - t_1)} \cdot (1 - e^{-\lambda \cdot t_{\text{real}}}) \cdot \varepsilon \cdot Y}
\end{equation}

\textbf{Výhody:}

\begin{itemize}
    \item Měření není ovlivněno asymetrií zdroje.
    \item vysoký účinný průřez pro absorpci neutronů na $^{55}$Mn je znám s vysokou přesností.
    \item Meto není citlivá na $\gamma$ záření, neboť neovlivňuje $^{55}$Mn.
    \item V lázni je jen jeden RN.
    \item Ne příliš dlouhý ale ani krátký poločas rozpadu a jednoduché přeměnové schéma vznikajícího RN činí tuto metodu nenáročnou na praktické měření.
\end{itemize}

\subsection{Metoda registrace doprovodných částic}

\begin{itemize}
    \item Vhodné pro zdroje neutronů založené na urychlování nabitých částic.
    \item Měření anbitých částic spojených s emisí neutronu.
    \item Potřeba tenkého terčíku -- vyloučení samoabsorpce.
    
\end{itemize}

Máme státní etalon emise neutronů -- manganová lázeň + scintilační detektor ve speciálním tubusu (nejistota 0,2 \%).

\subsection{Metody měření hustoty toku neutronů}

\begin{itemize}
    \item Hustota toku $\phi$ je stanovena pomocí reakční rychlosti $F$ = $N \cdot \sigma \cdot \phi$ $\rightarrow$ musím znát $\sigma$ s dostatečnou přesností.
    \item Obecně rozlišuji metody založené na:
    
    \begin{itemize}
        \item meření indukované aktivity,
        \item počítání reakčních produktů.
    \end{itemize}

    \item Hlavními problémy při měření jsou:
    
    \begin{itemize}
        \item narušení neutronového pole detektorem,
        \item anizotropie detektoru nebo nestejnoměrné oozáření,
        \item indukování nežádoucí radioaktivity.
    \end{itemize}

\end{itemize}

\textbf{Energie neutronů:}  

\begin{itemize}
    \item tepelné -- v rovnováze s prostředím 0,025 až 1 eV
    \item intermediální -- 0,5 eV až stovky keV
    \item rychlé -- jednotky MeV
\end{itemize}

\begin{figure}[H]
    \centering
    \includegraphics[width=0.8\linewidth, trim={1cm 12cm 1cm 12cm}, clip]{img/neutrony_energie.pdf}
    \caption{Rozdělení neutronů podle energie}
\end{figure}

\textbf{Měření indukované aktivity:}

\begin{itemize}
    \item Aktivita vzniklá v reakcích s neutrony $A(t)\,=\,n_{\text{R}} \cdot (1 - e^{-\lambda \cdot t})$.
    \item Může docházet k parazitním aktivačním reakcím $\rightarrow$ ovlivnění výsledků.
    \item Měření aktivity: koincidenční metodou $\beta-\gamma$ nebo počítání částíc v geometrii 4$\pi$.
    \item Rozsah většinou 10$^{10}$ až 10$^{22}$ cm$^{-2}$.
    \item Intermediální neutrony - rezonance $\rightarrow$ překryji detektor kadmiem $\rightarrow$ odfiltruji tepelné neutrony (kvůli ostrým maximům v $\sigma$).
    \item Rychlé neutrony $\rightarrow$ prahovými reakcemi.
\end{itemize}

\textbf{Počítání reakčních produktů:}

\begin{itemize}
    \item Většinou využívám (n,$\alpha$) nebo (n,p) případně (n,f) reakce.
    \item Fluence z počtu zaznamenaných částic (nutné určení účinnosti).
    \item Využívám u detektorů:
    \begin{itemize}
        \item scintilační,
        \item proporcionální počítače,
        \item štěpné komory,
        \item termoluminescenční,
        \item samonapájecí,
        \item fotografické emulze.
    \end{itemize}
\end{itemize}

\subsubsection{Detektory tepelných neutronů}

\begin{figure}[H]
    \centering
    \includegraphics[width=0.4\linewidth, trim={5cm 12cm 5cm 12cm},clip]{img/reakce_tepelne_neutrony.pdf}
    \caption{Detekce tepelných neutronů -- reakce}
\end{figure}

\begin{itemize}
    \item Plynové:
    \begin{itemize}
        \item heliové a borové proporcionální komory - plynová náplň $^{3}$He a $^{10}$BF$_3$ nebo pokrytí stěny $^{10}$B,
        \item štěpné ionizační komory - stěny poryté obohaceným uranem (velká kinetická energie fragmentů),
    \end{itemize}
    \item scintilační -- konverzní materiál součástí scintilátoru (např. $^{6}$LiI(Eu)),
    \item polovodičové -- konverzní vrstva na povrchu detektoru,
    \item termoluminiscenční -- LiF obohacený o $^{6}$Li,
    \item detektory stop v pevné fázi,
    \item aktivační -- (n,$\gamma$),
    \item samonapájecí.
\end{itemize}

\subsubsection{Detektory rychlých neutronů}

\begin{itemize}
    \item Dlouhý počítač -- založeno an principu moderace.
    \item Závislost odezvy na energii nalétávající částice je "po dlouhou dobu stejná" od určité energie.
\end{itemize}

\subsection{Spektrometrie neutronů}

\begin{itemize}
    \item Většinou se používají Bonnerovy sféry.
    \item Detektor tepelných neutronů se umístí do středu PE koule s různým průměrem -- ty slouží jako moderátor.
    \item Postupně se naberou spektra.
    \item Následně $\rightarrow$ unfolding = proces, kdy je uhodnuté spektrum neutornů" jako vstup a je následně stanoveno "skutečné".
    \item Je to vleklá magie, náročné na měření.
    \item Na reaktoru používám pouze jednu bonnerku jako měřidlo dávkového příkonu.
\end{itemize}

Máme státní etalon \textbf{fluence neutronů} a \textbf{hustoty toku neutronů}.

\begin{itemize}
    \item Etalon fluence:
    \begin{itemize}
        \item zdroje neutronů: $^{252}$Cf, 1E8 s$^{-1}$ a AmBe 2E10 s$^{-1}$ a generátor 14 MeV 1E9 až 1E10 s$^{-1}$,
        \item kalibrační lavice, Bonnerův spektrometr,
        \item měřidlo prostorového dávkového ekvivalentu neutronů (to samé na VR-1 -- ta těžká bílá koule).
    \end{itemize}

    \item Etalon hustoty toku tepelných neutronů:
    \begin{itemize}
        \item grafitová prizma -- RN zdroje vkládám do moderujícího prostředí (viz neutronová laborka),
        \item vytvářím pole tepelných neutronů pro potřeby kalibrace a ověřování měřidel.
    \end{itemize}
\end{itemize}

\textbf{DOPSAT ZPRACOVÁNÍ VÝSLEDKŮ A ZDROJE NEJISTOT (ASI V JINÉ PREZENTACI)}




\section[Jaderná bezpečnost \& ochrana do hloubky]{Základní principy jaderné bezpečnosti a ochrana do hloubky}


\section[Klasifikace událostí na jaderných zařízeních]{Klasifikace událostí na jaderných zařízeních a rozbor vybraných událostí}

\subsection{Klasifikace událostí na jaderných zařízeních}

\subsubsection{Provozní stavy reaktoru dle Vyhlášky č. 329/2017Sb.}

\begin{itemize}
	\item \textbf{Normální provoz} = V mezích limitů a podmínek, zahrnuje všechny stavy a operace plánovaného provozu.
	\item \textbf{Abnormální provoz} = Odchylky od normálního provozu, které jsou očekávané a nevedou k poškození SKK s vlivem na jadernou bezpečnost
	\item \textbf{Havarijní podmínky} = Stav jaderného zařízení, který není provozním stavem. Často se jedná o události negativně ovlivňující bezpečnost provozu a vedou k poškození zařízení a porušení limitů.
	
	\begin{itemize}
	    \item \textbf{Projektové nehody (DBA)} = Havrijní podmínky, při kterých správná funkce bezpečnonstních systému zajistí, že nedojde k překročení odpovídajících referenčních úrovní nebo limitů ozáření -$\rightarrow$ limitní nehody, poškození AZ a únik RA látek by měl zůstat v rámci limitů, např. LOCA.
        \item \textbf{Nadprojektové nehody (BDBA)} (v zákoně definováno jako rozšířené projektové podmínky) = Havarijní podmínky vyvolané scénáři závažnějšími než DBA, které jsou zohledněny při projektování jaderného zařízení $\rightarrow$ vážnější důsledky než DBA, např. LOCA + SBO.
        \item \textbf{Těžká havárie (SO)} = Havarijní podmínky, při kterých dochází k vážnému poškození jaderného paliva, a to vážným poškozením a nezvratnou ztrátou struktury aktivní zóny (AZ) jaderného reaktoru nebo systému pro skladování jaderného paliva poškozením palivových souborů v důsledku tavení jaderného paliva. => Jednoduše jsou havarijní podmínky, při kterých se vážně poškozuje palivo, ztráci se struktura AZ či systému pro skladování jaderného paliva, a to poškozením souborů v důsledku jeho tavení. Těžká havárie tedy nastává, až když se taví, přičemž platí, že je velmi nízká pravděpodobnost výskytu. 
	\end{itemize}

    \item \textbf{Postulovaná iniciační událost} = Odchylka od normálního provozu, která je náhodná, předpokládaná a je zahrnuta do projektových východisek a jejíž rozvoj může vést k abnormálnímu provozu nebo havarijním podmínkám.
\end{itemize}

\begin{figure}[h!]
    \centering
	\includegraphics[width=9cm]{img/jad-udalosti.png}
	\includegraphics[width=5cm]{img/jad-udalosti1.png}
\end{figure}

\begin{figure}[h!]
    \centering
    \includegraphics[width=\textwidth]{img/Stavy_JZ_cetnost.pdf}
\end{figure}

\subsubsection{Základní bezpečnostní funkce}

\begin{itemize}
    \item Umožňovat v případě potřeby okamžitě a bezpečně odstavit jaderný reaktor a udržovat jej v podkritickém stavu.
    \item Zabránit nekontrolovanému rozvoji štěpné řetězové reakce
    \item Fyzikálně znemožnit vznik kritického a nadkritického stavu mimo vnitřní prostor jaderného reaktoru.
    \item Zajišťovat odvod tepla vytvářeného jaderným palivem a technologickými systémy.
    \item Zajistit stínění a zabránit úniku radioaktivní látky a šíření ionizujícího záření do životního prostředí.
\end{itemize}

\begin{comment}
    
\subsubsection{Zajištění podkritičnosti}

\begin{itemize}

\item Lze uvažovat všechny existujı́cı́ bezpečnostnı́ systémy (v úrovni
ochrany do hloubky 3a)
\item Při využitı́ každého systému musı́ být uvažována nejvı́ce závažná jednoduchá porucha (navı́c k poruše v rámci metodiky)
\item Na začátku přechodového procesu se obvykle uplatnı́ systém rychlého odstavenı́ reaktoru
\item Následně udrženı́ stabilizovaného podkritického stavu
\end{itemize}
\subsubsection{Zajištění odvodu tepla}

\begin{itemize}
\item Zajištěnı́ technických kritériı́ přijatelnosti pro PIU kategorie AOO nebo DBA
\item Pro PIU AOO přı́snějšı́ kritéria přijatelnosti, za účelem zachovánı́ integrity pokrytı́ palivových elementů (kvůli vyššı́ četnosti výskytu)
\item Nesmı́ dojı́t k nepřijatelnému zvýšenı́ tlaku v I. okruhu nebo II.
okruhu
\item Důsledkem nedostatečného odvodu tepla z AZ může být až eventuelně uvolnění RA látek do I.O. či dále až do ŽP.
\end{itemize}

\subsubsection{Zadržení radioaktivních látek}
\begin{itemize}
\item Neporušenı́ fyzických bariér (s výjimkou bariér porušených
samotnou PIU) se v deterministických analýzách DBE prokazuje
splněnı́m odpovı́dajı́cı́ch radiačnı́ch a technických kritériı́
přijatelnosti
\item Předpokládá, že KNTM je při vzniku PIU a v průběhu jejı́ho
rozvoje hermeticky uzavřen
\item Udrženı́ projektem stanovené těsnosti KTMT musı́ být prokázáno
během rozvoje DBE (i při tlakovému a teplotnı́mu namáhánı́ a
úniku radioaktivnı́ch látek z I.O)
\item KTMT je chráněn proti ztrátě základnı́ bezpečnostnı́ funkce
působenı́m přetlaku systémem řı́zenı́ tlaku a teploty
\end{itemize}


\subsubsection{Zařízení pro jadernou bezpečnost}
\begin{itemize}
    \item zařízení nedůležitá pro JB
    \item zařízení důležitá pro JB = vybraná zařízení - systém nebo komponenta nebo konstrukce důležitá pro jadernou bezpečnost a má vliv na plnění bezpečnostních funkcí, její selhání může vést k ozáření personálu nebo obyvatelstva
    \item zařízení s vlivem na JB které nejsou vybraným zařízením
    \item zařízení pro prevenci rozvoje iniciačních událostí - prostředky pro omezení důsledků selhání vybraných zařízení, prevence rozvoje havarijních podmínek
\end{itemize}

\end{comment}


\textit{Asi by bylo dobré zmínit, že hodnocení jaderné bezpečnosti provádíme buď deterministickými nebo pravděpodobnostními hodnoceními.}

Deterministický přístup:

\begin{itemize}
    \item Definice iniciačních událostí $\rightarrow$ musím aplikovat specifické požadavky.
    \item Určení dostupnosti SKK, okrajových a počátečních podmínek (jednoduchá porucha, zvládání nehody pouze bezpečnostními systémy).
    \item Výběr výpočetního kódu.
    \item Zhodnocení výsledků.
\end{itemize}

Pravděpodobnostní přístup (PSA):

\begin{itemize}
    \item Vychází z deterministických analýz, ale beru v úvahu pravděpodobnost.
    \item Využívám při risk managementu, mám pravděpodobnost, že se něco stane, mohu optimalizovat vůči přínosu.
    \item CDF, pravděpodobnost úniku RA, pravděpodobnost účinku na veřejnost.
\end{itemize}

\subsubsection{Klasifikace událostí}

\begin{itemize}
	\item Fyzikální přístup
	
	\begin{itemize}
		\item Havárie vyvolané kladnou změnou reaktivity (RIA)
		\item Havárie se ztrátou chladiva (LOCA)
		\item Havárie v systému odvodu tepla
		\item Ostatní havárie
		\item Vnější vlivy
	\end{itemize}

	\item Podle iniciačních událostí
	\item stupnice INES
\end{itemize}


\textbf{Havárie vyvolané kladnou změnou reaktivity}

Kladná změna reaktivity v kritickém reaktoru vede ke zvýšení výkonu. Ten závisí na rychlosti a velikosti vložené kladné reaktivity, účinku zpětných vazeb a zásahu regulačního systému. Problém je kritičnost na okamžitých neutronech.

Vliv má na jakém výkonu je reaktor před vložením kladné reaktivity:

\begin{itemize}
    \item Nízký výkon (spouštění) -- nepůsobí zpětné vazby, chování systému je dáno dynamikou nulového reaktoru.
    \item Vysoký (nominální) výkon -- zafungují zpětné vazby, které omezí nárůst výkonu (Doppler), ale parametry reaktoru už jsou blízko limitních hodnot (maximální dovolené teploty) $\rightarrow$ často vede na lokální poškození paliva. Proutek to může vydržet ale tableta se mohla lokálně natavit.
\end{itemize}

Možné  scénáře:

\begin{itemize}
    \item Nekontrolované vysouvání regulačních orgánů = Závisí na maximální rychlosti vysouvání RO a maximální vázané reaktivitě, přechodový proces závisí na výkonu reaktoru (jiné chování u nulového reaktoru a reaktoru na výkonu), nárůst výkonu je omezen zpětnými vazbami a zpravidla nepředstavuje závažné ohrožení.
    \item Vystřelení regulačního orgánu = Teoreticky může nastat při prasknutí pouzdra pohonu RO, v důsledku okamžitého poklesu tlaku v pouzdře dojde k vystřelení tyče z AZ, rychlý proces, ale celková reaktivita je menší než u předchozího, je spojeno s únikem chladiva.
    \item Náhlé uvolnění usazenin absorpčního charakteru (bór) v konstrukci (bór se v průběhu provozu usazuje na konstrukčních částech), při jejich náhlém uvolnění = Kladná změna, rychlý přechodový proces jako vystřelení tyče, závisí na maximální uvolněné reaktivitě.
    \item Nekontrolované snižování koncentrace rozpustných absorbátorů = Chybné snížení koncentrace nebo připojení trasy s čistým kondenzátem, změny koncentrace jsou pomalé $\rightarrow$ prostor pro vyřešení situace nebo odstavení reaktoru, významná změna reaktivity je v řádu hodin.
    \item Chybné zavezení paliva = Dosažení nadkritického stavu z důvodu špatně zavezeného čerstvého PS, vylučuje se zvýšením koncentrace kyseliny borité a administrativní opatření při zavážení PS.
    \item Vtok studené vody do AZ = Uvolnění kladné reaktivity díky zápornému teplotnímu koeficientu reaktivity chladiva (hodnota závisí na koncentraci kyseliny borité).
    \item Vliv tlakových změn = Vliv dutinového koeficientu reaktivity, problém je u vody, která je blízko meze sytosti a v části chladiva probíhá bublinkový var, při záporném dutinovém koeficientu zvýšení tlaku vede ke snížení podílu páry a uvolnění kladné reaktivity.
\end{itemize}

\textbf{Havárie se ztrátou chladiva}

Snížení nebo úplné přerušení chlazení primárního okruhu (SB nebo LB LOCA). Jde o havárie typu LOCA (Loss of Coolant Accident).

Možné  scénáře:

\begin{itemize}
    \item Prasknutí hlavního potrubí IO = LB LOCA, prasknutí potrubí se zachováním plného výtokového průřezu.
    \item Prasknutí potrubí s malým nebo středním únikem = SB LOCA, vyšší pravděpodobnost výskytu a odlišný průběh přechodového procesu, může být horší než u LB LOCA.
    \item Prasknutí trubky v PG = Patří do skupiny progresivních poruch, tj. poruchy, které jsou malé a lokalizované, ale postupně se rozrůstají do vážných rozměrů, šíření poruch ve svazku trubek PG.
\end{itemize}

\textbf{Havárie v systému odvodu tepla}

Jde o poruchy vedoucí ke snížení odvodu tepla z reaktoru. Týkají se primárního i sekundárního okruhu, i odstaveného reaktoru (zbytkový výkon).

Možné  scénáře:

\begin{itemize}
    \item Selhání HCČ = Snížení průtoku vede ke zvýšení teploty a tlaku chladiva v IO $\rightarrow$ ROR, pokud nedojde k ROR otevírá se PV na KO a únik do barbotážního systému, průběh závisí na koncepci IO, počtu vypadlých HCČ, setrvačnosti čerpadel, rychlosti odstavení reaktoru, zásahu obsluhy, ...
    \item Nestability proudění chladiva v AZ = Snížení průtoku v některých kanálech AZ nebo skupině kanálů vede k narušení hydrodynamiky AZ a přehřátí části PP, příčinou havárie může být částečné zablokování průtoku chladiva CP (cizí předmět) nebo vznik varu v některém PS (vyšší hydraulický odpor).
    \item Selhání dodávky napájecí vody do PG = hodně možností, tj. vyšší relativní četnost poruchy (hlavní příčiny - selhání napájecích čerpadel, výpadek napětí v síti vlastní spotřeby, prasknutí potrubí napájecí vody), havarijní analýzy řeší následek prasknutí hlavního napájecího kolektoru s oboustranným výtokem (Loss of Feedwatter - LOFW) $\rightarrow$ uzavření ventilu na napájecí trase kvůli zamezení výtoku vody z PG, v PG roste tlak, snižuje se hladina na sekundární straně, v IO roste teplota a tlak.
    \item Prasknutí HPK nebo hlavního parovodu = Nejhorší havárie na parním potrubí IIO, důsledkem je prudký pokles tlaku a teploty na teplotu sytosti v IIO, pokles tlaku v IIO vyvolá signál pro zvýšení výkonu reaktoru a zvýšení dodávek napájecí vody do PG, to vede k snížení teploty, objemu a tlaku v IO a dalšímu zvýšení výkonu díky zpětným vazbám.
    \item Selhání hlavních systémů odvodu tepla v kondenzátoru = Poruchy chlazení kondenzátorů TG vedou k odstavení TG.
\end{itemize}

\textbf{Ostatní havárie}

Zahrnuje situace, které mohou nastat při odstávkách nebo v systémech, které přímo nesouvisí s provozem reaktoru. Tyto události mohou vést k rozsáhlému úniku RA látek.

Možné scénáře:

\begin{itemize}
    \item Havárie při manipulaci s palivem = Čerstvé (havárie s kritičností) nebo ozářené palivo (kritičnost a ztráta chlazení v bazénu VJP).
    \item Havárie systému zpracování RAO = Události při skladování, přepracování a ukládání RAO.
\end{itemize}

\textbf{Vnější vlivy}

\begin{itemize}
    \item Velmi málo pravděpodobné povětrnostní podmínky a živelní pohromy.
    \item Vlivy vyplývající z lidské činnosti.
    \item Zemětřesení.
    \item Záplavy, průtrž mračen, vítr.
    \item Požáry.
    \item Tlaková vlna (důsledek výbuchu).
    \item Pád letadla.
    \item Sabotáže.
\end{itemize}

\subsection{Nehody}

\subsubsection{Projektové nehody (DBA)}

DBA přestavují základní sadu událostí, které musí být pokryty bezpečnostními systémy, tak aby nedošloze k překročení odpovídajících referenčních úrovní nebo limitů ozáření.

\begin{itemize}
    \item Zvládnutí DBA je nezbytné pro vyhodnocení přijatelnosti návrhu reaktoru.
    \item Při DBA je jedna nebo více bezpečnostních funkcí je napadena.
    \item Pro zvládnutí události je nutný zásah bezpečnostních systémů.
    \item DBA by neměly mít dopad na okolí elektrárny a veřejnost.
    \item Zpravidla se během provozu elektrárny nevyskytne.
    \item Frekvence výskytu $10^{-4}$ až $10^{-2}$ za 1 rok.
\end{itemize}

\textbf{Hodnocení bezpečnosti:}

\begin{itemize}
    \item Proces komplexního deterministického hodnocení a PSA.
    \item Hodnocení prokazuje, že JZ plní v rozsahu projektových východisek požadavky na JB a RO při normálním a abnormálním provozu.
    \item Plnění bezpečnostních funkcí je v analýzách zajišťováno jen bezpečnostními systémy.
    \item Musí být zahrnuto kritérium jednoduché poruchy (uvažuje se nejzávažnější možná porucha).
    \item Dopad PIE na bezpečnostní systémy je konzervativně volen jako nejhorší $\rightarrow$ nejméně příznivý průběh události.
    \item Zohledňují se nejistoty a výrobní tolerance.
    \item Deterministické hodnocení -- pro všechny PIE $\rightarrow$ následky nehody.
    \item Pravděpodobnostní hodnocení $\rightarrow$ pravděpodobnost CDF a LRF.
\end{itemize}

\subsubsection{Nadprojektové nehody (BDBA)}

Jde o PIE vedoucí k DBA spolu s dalším selháním bezpečnostních systémů $\rightarrow$ vyvolané scénáři závažnějšími než základní projektová nehoda. Např. ztráta ECCS během LOCA, ATWS (událost spojená s neodstavením reaktoru při vyžádání jeho odstavení, např. selhání pádu tyče), totální ztráta napájecí vody, SBO.

\begin{itemize}
    \item V odezvě na BDBA se zpravidla objevují akce operátorů na zmírnění důsledků nehody:
    \begin{itemize}
        \item Odpouštění a doplňování chladiva v IO (snižování tlaku přes KO, doplňování vody přes ECCS).
        \item Odpouštění a doplňování chladiva v IIO (PV, havarijní napájení PG).
        \item Hermetizace (izolace) KNTM.
        \item Odtlakování KNTM (ventilace KNTM).
    \end{itemize}
\end{itemize}

\textbf{Hodnocení bezpečnosti:}

\begin{itemize}
    \item Na základě deterministického a pravděpodobnostního hodnocení vybrány bezpečnostně nejvýznamnější BDBA události.
    \item Jedná se o PIE vedoucí k DBA + rozšířené podmínky, např. nedostupnost ECCS během LOCA, SBO.
    \item Velmi málo pravděpodobné události.
    \item V bezpečnostních analýzách se používají méně konzervativní kritéria.
    \item Mohou být použity realistické předpoklady $\rightarrow$ best-estimate přístup -- neuvažuje se kritérium jednoduché poruchy, počítá se se zásahem systémů, které nejsou bezpečnostní.
    \item Hodnotí se průběhy i radiační následky (identifikace nutných opatření, podklady pro návody zvládání RMU a výcvik obsluhy, ...)
\end{itemize}

\subsubsection{Těžké havárie (SA)}

Důsledkem je poškození AZ, roztavení paliva a případně i významný únik RA látek.

\begin{itemize}
    \item Procesy v RN: poškození paliva (oxidace Zr při vysokých teplotách, 1200 $^\circ$C), parní exploze, porušení integrity.
    \item Poškození RN.
    \item Procesy mimo RN: zahřívání povrchů, exploze vodíku, interakce betonu s kóriem.
\end{itemize}

\subsection{Mezinárodní stupnice -- INES}

= The International Nuclear Event Scale

\begin{itemize}
    \item Vyvinuta v roce 1990 (IAEA).
    \item Určena pro rychlou orientaci veřejnosti. Je to komunikační nástroj, který usnadňuje komunikaci mezi odbornou a laickou veřejností a médii.
    \item Používá se pro hodnocení událostí na všech zařízeních souvisejících s jaderným průmyslem.
    \item Dodatečně rozšířena pro potřeby událostí spojených s dopravou, skladováním a použitím radioaktivních látek a zdrojů záření.
    \item Zahrnuje ztrátu nebo krádež RA zářičů, nález opuštěných zářičů či neplánované ozáření osob při regulovaných dozorovaných činnostech.
    \item Neslouží pro srovnávání bezpečnosti jednotlivých elektráren.
    \item Neslouží pro hodnocení zabezpečení zdrojů IZ.
    \item Nevztahuje se na události spojené pouze s technickou bezpečností a pro události, které nemají vztah k jaderné nebo radiační bezpečnosti, tj. únik chemikálií, zásah elektrickým proudem, události ovlivňující pohotovost turbíny nebo generátoru, pokud nebyl ovlivněn výkon reaktoru. Jedná se tedy spíše o radiologickou než technologickou stupnici.
\end{itemize}

\textbf{Klasifikace do sedmi stupňů:}

\begin{itemize}
	\item Stupeň 4-7 -- havárie (accidents).
	\item Stupeň 1-3 -- nehody (incidents).
	\item Stupeň 0/pod stupnicí -- události bez vlivu na bezpečnost.
\end{itemize}
	
Bezpečnostní význam mají události 2. a vyššího stupně. Většina hlášených událostí je do 3. stupně.

\begin{figure}[h!]
	\centering
	\includegraphics[width=8cm]{img/ines.png}
	\caption{INES}
	\label{fig:ines} 
\end{figure}

\begin{itemize}
	\item[0)] \textbf{Událost bez vlivu na bezpečnost}.
	\item[1)] \textbf{Anomálie} = technická porucha nebo odchylka od schváleného režimu, ale se zbývající významnou hloubkovou ochranou.
	\item[2)] \textbf{Nehoda} = nehoda s významným selháním bezpečnostních opatření, ale se zbývající dostatečnou hloubkovou ochranou k vypořádání se dodatečnými poruchami.
	\item[3)] \textbf{Vážná nehoda} = významné vnitřn.í poškození ovšem bez nutnosti vnějšího zásahu, další porucha bezpečnostních systémů by mohla vést k havarijním podmínkám.
	\item[4)] \textbf{Havárie bez vážnějšího rizika} = havárie vně zařízení, bez významného vlivu na okolí, významné poškození zařízení (např. částečné tavení AZ), ozáření jednotlivců z obyvatelstva na úrovni ročních limitů pro obyvatelstvo, ozáření pracovníků na úrovni možných časných úmrtí.
	\item[5)] \textbf{Havárie s rizikem} = havárie vně zařízení, těžké poškození jaderného zařízení (havárie s kritičností, velký požár a exploze), únik významného množství radioaktivity, aplikace opatření pro snížení rizika poškození zdraví u obyvatelstva.
	\item[6)] \textbf{Těžká havárie} = velký únik radioaktivity do životního prostředí, plná aktivace opatření pro snížení pravděpodobnosti zdravotních následků u obyvatelstva.
	\item[7)] \textbf{Velmi těžká havárie} = únik velmi velkého množství radioaktivity, možnost akutních zdravotních účinků, dlouhodobé poškození životního prostředí, vliv přesahuje hranice států.
\end{itemize}

\textbf{Kritéria hodnocení dle INES:}

Každá událost se hodnotí vzhledem ke třem kritériím, přičemž výsledné hodnocení je dáno nejvyšším stupněm dle všech tří kritérií.

\begin{itemize}
    \item Obyvatelé a ŽP (stupně 2-7):
    Je založeno na dávkách obyvatel, tj. obdržená dávka a počet ozářených osob. Vzhledem k použití havarijních opatření můžou být obdržené dávky velmi nízké
    \item Radiační bariéry a opatření (stupně 2-5):
    Je založeno na množství uvolněných radioaktivních látek. Pro zahrnutí širokého spektra radioaktivních izotopů, které se mohou uvolnit do ŽP se používá koncept radiačního ekvivalentu $^{131}$I. Jsou definovány konverzní faktory pro jednotlivé izotopy.
    \item DID (stupně 1-3).
\end{itemize}

INES poskytuje rychlou zprávu o významu události $\rightarrow$ klasifikace na základě prvotního odhadu, avšak hodnocení se může měnit podle vývoje situace. Výsledné hodnocení je dáno nejvyšším stupněm podle všech tří kritérií! Ve výsledku se bere nejvyšší stupeň, a to z důvodu, že u jednoho kritéria může být výsledkem stupeň 1, ale z pohledu jiného může být mnohem výše.

\textbf{Dávky pro jednotlivce:}

Kritérium obdržené dávky je nejpřímější kritériem. Definice se vztahují k dávkám, které byly obdrženy, nebo velmi pravděpodobně mohly být obdrženy (tj. zvážit výskyt obdržených dávek, o kterých se nevědělo). Hodnocení vychází z ozáření jedné osoby, ozáření více osob je důvodem ke zvýšení hodnocení až o dva stupně.

\begin{itemize}
    \item Stupeň 4: výskyt smrtelného deterministického účinku. 
    Pravděpodobný výskyt smrtelného deterministického účinku v důsledku celotělového ozáření (absorbovaná dávka řádu několika Gy).
    \item Stupeň 3: výskyt nebo pravděpodobný výskyt deterministických účinků, při nichž nedojde ke smrti.
    Ozáření vedoucí k efektivní dávce vyšší než je 10x stanoveného ročního limitu pro pracovníky se zářením 20 mSv.
    \item Stupeň 2: ozáření jednoho obyvatele vedoucí k efektivní dávce přesahující 10 mSv.
    Ozáření pracovníka přesahující stanovené roční limity.
    \item Stupeň 1: ozáření jednoho obyvatele přesahující roční dávkové limity.
    Kumulativní ozáření pracovníka nebo obyvatele přesahující stanovené roční limity.
\end{itemize}

\subsubsection{Příklady událostí}

\begin{itemize}
	\item[1)] Porušení technických podmínek, nehody bez přímých důsledků, které odhalí nedostatky v organizačním systému nebo kultuře bezpečnosti, menší defekty v potrubí.
	\item[2)] Ishikava (Japonsko), Mihama (Japonsko, 1991), Atucha (Argentina), Forsmark (švédsko), Cadarache (Francie), Pickering A-B (Kanada, 2003)
	\item[3)] Selafield (UK), Paks (Maïarsko, 2002)), Vandellos (španělsko, 1989), Davis Besse-1, (USA, 2002)
	\item[4)] Jaslovské Bohunice (Slovenská rep.), Windscale Pile (VB, 1973), Saint Laurent (Francie, 1980), Tokaimura (Japonsko, 1999)
	\item[5)] TMI2 (USA, 1979), Chalk River (Kanada), Windscale Pile (VB, 1957)
	\item[6)] Kyštym (Majak) (Rusko, 1957) -- obohacovací závod
	\item[7)] Chernobyl (Ukrajina, 1986), Fukushima (Japonsko, 2011)
\end{itemize}

\begin{figure}[h!]
    \centering
    \includegraphics[width=0.9\textwidth]{img/INES_tab.pdf}
\end{figure}
\section[Legislativa jaderné bezpečnosti]{Postavení provozovatele, státního dozoru a IAEA v jaderné bezpečnosti, legislativní rámec jaderné bezpečnosti}

\end{document}
